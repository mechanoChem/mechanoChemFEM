\hypertarget{growth_Introduction}{}\section{Introduction}\label{growth_Introduction}
The Allen Cahn equation is\+: \[ \frac{\partial C}{\partial t}=-M\mu\\ \mu=\frac{\partial g}{\partial C}-\nabla\cdot k\nabla C \] where $g$ is a non-\/convex, \`{}`homogeneous'\textquotesingle{} free energy density function, whose form has been chosen \[ g(C)=\omega(C-C_\alpha)^2(C-C_\beta)^2 \] The boundary condition is \[ \nabla \mu\cdot\boldsymbol{n}=0 \text{ on }\Gamma \] The double-\/well non-\/convex free energy density function, $g(C)$, drives segregation of the system into two distinct types.

Before implementing the code, we rewrite the equation in the form of standard diffusion-\/reaction equations\+: \[ \frac{\partial C}{\partial t}+\nabla\cdot\boldsymbol{j}=r\\ \boldsymbol{j}=-Mk\nabla C\\ r=-M\frac{\partial g}{\partial C}=-2M\omega(C-C_\alpha)(C-C_\beta)(2C-C_\alpha-C_\beta) \]\hypertarget{growth_imple}{}\section{Implementation}\label{growth_imple}
We first define the single scalar primary variable\+: 
\begin{DoxyCode}{0}
\DoxyCodeLine{  std::vector<std::vector<std::string> > primary\_variables(1);        }
\DoxyCodeLine{primary\_variables[0].push\_back(\textcolor{stringliteral}{"mu"}); primary\_variables[0].push\_back(\textcolor{stringliteral}{"component\_is\_scalar"});}
\end{DoxyCode}
 and setup the order of basis function for it\+: 
\begin{DoxyCode}{0}
\DoxyCodeLine{\textcolor{keywordtype}{int} number\_domain=1;}
\DoxyCodeLine{\textcolor{keywordtype}{int} diff\_degree=1;}
\DoxyCodeLine{std::vector<std::vector<int> > FE\_support(number\_domain);\textcolor{comment}{// store order of polynomial basis functions, 0 means FE\_Nothing   }}
\DoxyCodeLine{FE\_support[0].push\_back(diff\_degree);}
\end{DoxyCode}
 Before launching the \href{../html/classinit_bound_val_probs.html}{\texttt{ init\+Bound\+Val\+Probs}}, we need to initialize the {\bfseries{Parameter\+Handler}} and declare all paramters we may use\+: 
\begin{DoxyCode}{0}
\DoxyCodeLine{ParameterHandler params;}
\DoxyCodeLine{params.enter\_subsection(\textcolor{stringliteral}{"Parameters"});  }
\DoxyCodeLine{params.declare\_entry(\textcolor{stringliteral}{"omega"},\textcolor{stringliteral}{"0"},Patterns::Double() );}
\DoxyCodeLine{params.declare\_entry(\textcolor{stringliteral}{"c\_alpha"},\textcolor{stringliteral}{"0"},Patterns::Double() );}
\DoxyCodeLine{\textcolor{comment}{//... more parameters }}
\DoxyCodeLine{params.leave\_subsection();  }
\end{DoxyCode}
 Now we just need to have class inherited from \href{../html/classinit_bound_val_probs.html}{\texttt{ init\+Bound\+Val\+Probs}} class, and overload the \href{../html/classinit_bound_val_probs.html\#ac8f2c3e2a1040c70b709900dc3dfdaea}{\texttt{ get\+\_\+residual()}} function\+: 
\begin{DoxyCode}{0}
\DoxyCodeLine{\textcolor{keyword}{template} <\textcolor{keywordtype}{int} dim>}
\DoxyCodeLine{\textcolor{keyword}{class }AllenCahn: \textcolor{keyword}{public} initBoundValProbs<dim>}
\DoxyCodeLine{\{}
\DoxyCodeLine{    \textcolor{keyword}{public}:}
\DoxyCodeLine{        AllenCahn(std::vector<std::vector<std::string> > \_primary\_variables, std::vector<std::vector<int> > \_FE\_support, ParameterHandler\& \_params);}
\DoxyCodeLine{        \textcolor{comment}{//this is a overloaded function }}
\DoxyCodeLine{        \textcolor{keywordtype}{void} get\_residual(\textcolor{keyword}{const} \textcolor{keyword}{typename} hp::DoFHandler<dim>::active\_cell\_iterator \&cell, \textcolor{keyword}{const} FEValues<dim>\& fe\_values, Table<1, Sacado::Fad::DFad<double> >\& R, Table<1, Sacado::Fad::DFad<double>>\& ULocal, Table<1, double >\& ULocalConv);}
\DoxyCodeLine{        ParameterHandler* params;       }
\DoxyCodeLine{\};}
\end{DoxyCode}
 In the overloaded {\bfseries{get\+\_\+residual}} function, we define the residual for our problem. As our equation is standrad diffusion-\/reaction, we can simiply use the pre-\/defined model\+: 
\begin{DoxyCode}{0}
\DoxyCodeLine{this->ResidualEq.residualForDiff\_ReacEq(fe\_values, 0, R, mu, mu\_conv, j\_mu, rhs\_mu);}
\end{DoxyCode}
 Before we call these two functions, we need the flux and reactions term, we need to first evaluate the values of the primary field and the spatial gradient. We also need to evaluate the value of primary field at previous time step for the Backward Euler time scheme\+: 
\begin{DoxyCode}{0}
\DoxyCodeLine{dealii::Table<1,double>  mu\_conv(n\_q\_points);}
\DoxyCodeLine{dealii::Table<1,Sacado::Fad::DFad<double> >  mu(n\_q\_points);}
\DoxyCodeLine{dealii::Table<2,Sacado::Fad::DFad<double> >  mu\_grad(n\_q\_points, dim);}
\DoxyCodeLine{}
\DoxyCodeLine{evaluateScalarFunction<double,dim>(fe\_values, 0, ULocalConv, mu\_conv);}
\DoxyCodeLine{evaluateScalarFunction<Sacado::Fad::DFad<double>,dim>(fe\_values, 0, ULocal, mu);    }
\DoxyCodeLine{evaluateScalarFunctionGradient<Sacado::Fad::DFad<double>,dim>(fe\_values, 0, ULocal, mu\_grad);}
\DoxyCodeLine{}
\DoxyCodeLine{\textcolor{comment}{//evaluate diffusion and reaction term}}
\DoxyCodeLine{dealii::Table<1,Sacado::Fad::DFad<double> > rhs\_mu(n\_q\_points);}
\DoxyCodeLine{dealii::Table<2,Sacado::Fad::DFad<double> > j\_mu(n\_q\_points, dim);}
\DoxyCodeLine{}
\DoxyCodeLine{j\_mu=table\_scaling<Sacado::Fad::DFad<double>, dim>(mu\_grad,-kappa*M);}
\DoxyCodeLine{}
\DoxyCodeLine{\textcolor{keywordflow}{for}(\textcolor{keywordtype}{unsigned} \textcolor{keywordtype}{int} q=0; q<n\_q\_points;q++)\{}
\DoxyCodeLine{     rhs\_mu[q]=-M*(2*omega*(mu[q]-c\_alpha)*(mu[q]-c\_beta)*(2*mu[q]-c\_alpha-c\_beta));}
\DoxyCodeLine{ \}}
\end{DoxyCode}
 The last thing we need to define is the initial condition, we can simpily overload the \href{../html/class_initial_conditions.html\#a369cea7ba74f8cd0a6ca12e0c164ff74}{\texttt{ value()}} function of the \href{../html/class_initial_conditions.html}{\texttt{ Initial\+Conditions }} class to define the initial condition for single scalar field. 
\begin{DoxyCode}{0}
\DoxyCodeLine{\textcolor{keywordtype}{double} InitialConditions<dim>::value(\textcolor{keyword}{const} Point<dim>   \&p, \textcolor{keyword}{const} \textcolor{keywordtype}{unsigned} \textcolor{keywordtype}{int}  component)\textcolor{keyword}{ const}\{}
\DoxyCodeLine{  \textcolor{keywordflow}{return} 0.5 + 0.2*(static\_cast <double> (rand())/(static\_cast <double>(RAND\_MAX))-0.5); }
\DoxyCodeLine{\}}
\end{DoxyCode}
\hypertarget{growth_results}{}\section{Results}\label{growth_results}
 

The results are generated using paramters shown below. The complete implementaion can be found at \href{https://github.com/mechanoChem/mechanoChemFEM/tree/example/Example3_Allen-Cahn}{\texttt{ Github}}.


\begin{DoxyCode}{0}
\DoxyCodeLine{\textcolor{preprocessor}{\#parameters file}}
\DoxyCodeLine{}
\DoxyCodeLine{subsection Problem}
\DoxyCodeLine{set print\_parameter = \textcolor{keyword}{true}}
\DoxyCodeLine{}
\DoxyCodeLine{set dt = 1}
\DoxyCodeLine{set totalTime = 50}
\DoxyCodeLine{set current\_increment = 0}
\DoxyCodeLine{set off\_output\_index=0}
\DoxyCodeLine{set current\_time = 0}
\DoxyCodeLine{set resuming\_from\_snapshot = \textcolor{keyword}{false}}
\DoxyCodeLine{}
\DoxyCodeLine{set output\_directory = output/}
\DoxyCodeLine{set snapshot\_directory = snapshot/}
\DoxyCodeLine{}
\DoxyCodeLine{\textcolor{preprocessor}{\#FEM}}
\DoxyCodeLine{set volume\_quadrature = 3 }
\DoxyCodeLine{set face\_quadrature = 2 }
\DoxyCodeLine{end}
\DoxyCodeLine{}
\DoxyCodeLine{subsection Geometry}
\DoxyCodeLine{set X\_0 = 0}
\DoxyCodeLine{set Y\_0 = 0}
\DoxyCodeLine{set Z\_0 = 0}
\DoxyCodeLine{set X\_end = 1 }
\DoxyCodeLine{set Y\_end = 1}
\DoxyCodeLine{set Z\_end = 2.0 \#no need to 2D}
\DoxyCodeLine{}
\DoxyCodeLine{set element\_div\_x=100}
\DoxyCodeLine{set element\_div\_y=100}
\DoxyCodeLine{set element\_div\_z=5 \#no need to 2D}
\DoxyCodeLine{end}
\DoxyCodeLine{}
\DoxyCodeLine{subsection Parameters}
\DoxyCodeLine{set omega = 0.8}
\DoxyCodeLine{set c\_alpha = 0.2}
\DoxyCodeLine{set c\_beta = 0.8}
\DoxyCodeLine{set kappa = 0.001}
\DoxyCodeLine{set M=1}
\DoxyCodeLine{end}
\DoxyCodeLine{                        }
\DoxyCodeLine{\textcolor{preprocessor}{\#}}
\DoxyCodeLine{\textcolor{preprocessor}{\# parameters reserved for deal.ii first level code:}}
\DoxyCodeLine{\textcolor{preprocessor}{\#nonLinear\_method : classicNewton}}
\DoxyCodeLine{\textcolor{preprocessor}{\#solver\_method (direct) : PETScsuperLU, PETScMUMPS}}
\DoxyCodeLine{\textcolor{preprocessor}{\#solver\_method (iterative) : PETScGMRES PETScBoomerAMG}}
\DoxyCodeLine{\textcolor{preprocessor}{\#relative\_norm\_tolerance, absolute\_norm\_tolerance, max\_iterations}}
\DoxyCodeLine{\textcolor{preprocessor}{\#}}
\DoxyCodeLine{subsection Nonlinear\_solver}
\DoxyCodeLine{        set nonLinear\_method = classicNewton}
\DoxyCodeLine{        set relative\_norm\_tolerance = 1.0e-12}
\DoxyCodeLine{        set absolute\_norm\_tolerance = 1.0e-12}
\DoxyCodeLine{        set max\_iterations = 10}
\DoxyCodeLine{end}
\DoxyCodeLine{                        }
\DoxyCodeLine{subsection Linear\_solver}
\DoxyCodeLine{        set solver\_method = PETScsuperLU}
\DoxyCodeLine{        set system\_matrix\_symmetricFlag = \textcolor{keyword}{false} \# \textcolor{keywordflow}{default} is \textcolor{keyword}{false}}
\DoxyCodeLine{end}
\end{DoxyCode}
 