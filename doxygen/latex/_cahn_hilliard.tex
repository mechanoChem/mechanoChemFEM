\hypertarget{growth_Introduction}{}\section{Introduction}\label{growth_Introduction}
The Cahn-\/\-Hilliard equation is\-: \[ \frac{\partial C}{\partial t}=\nabla\cdot(M\nabla\mu)\\ \mu=\frac{\partial g}{\partial C}-\nabla\cdot k\nabla C \] where $g$ is a non-\/convex, ``homogeneous'' free energy density function, whose form has been chosen \[ g(C)=\omega(C-C_\alpha)^2(C-C_\beta)^2 \] The boundary condiiton is \[ \nabla\mu\cdot\boldsymbol{n}=0; \nabla C\cdot\boldsymbol{n}=0 \text{ on }\Gamma \] The double-\/well non-\/convex free energy density function, $g(C)$, drives segregation of the system into two distinct types.

Attention is called to the well-\/known fourth-\/order nature of this partial differential equation in the concentration $C$. The polynomial basis can only achieve C0 continuity across the element. To overcome this difficuity, we split the equation into two equations\-: \[ \frac{\partial C}{\partial t}+\nabla\cdot(-M\nabla\mu)=0\\ k\nabla^2 C=\frac{\partial g}{\partial C}-\mu \] The first equation is diffusion equation, and the second one the Possion equation. \hypertarget{growth_imple}{}\section{Implementation}\label{growth_imple}
We first define the two scalar primary variables\-: 
\begin{DoxyCode}
std::vector<std::vector<std::string> > primary\_variables(2);        
      primary\_variables[0].push\_back(\textcolor{stringliteral}{"c1"}); primary\_variables[0].push\_back(\textcolor{stringliteral}{"component\_is\_scalar"});
      primary\_variables[1].push\_back(\textcolor{stringliteral}{"mu"}); primary\_variables[1].push\_back(\textcolor{stringliteral}{"component\_is\_scalar"});
\end{DoxyCode}
 and we solve both species in one domain. We define the domain and basis order for each primal variables\-: 
\begin{DoxyCode}
\textcolor{keywordtype}{int} number\_domain=1;
\textcolor{keywordtype}{int} diff\_degree=1;
std::vector<std::vector<int> > FE\_support(number\_domain);\textcolor{comment}{// store order of polynomial basis functions, 0
       means FE\_Nothing   }
FE\_support[0].push\_back(diff\_degree);
FE\_support[0].push\_back(diff\_degree);
\end{DoxyCode}
 Before launching the \href{../html/classinit_bound_val_probs.html}{\tt init\-Bound\-Val\-Probs}, we need to initialize the {\bfseries Parameter\-Handler} and declare all paramters we may use\-: 
\begin{DoxyCode}
ParameterHandler params;
params.enter\_subsection(\textcolor{stringliteral}{"Concentration"});   
params.declare\_entry(\textcolor{stringliteral}{"omega"},\textcolor{stringliteral}{"0"},Patterns::Double() );
params.declare\_entry(\textcolor{stringliteral}{"c\_alpha"},\textcolor{stringliteral}{"0"},Patterns::Double() );
\textcolor{comment}{//... more parameters }
params.leave\_subsection();  
\end{DoxyCode}
 Now we just need to have class inherited from \href{../html/classinit_bound_val_probs.html}{\tt init\-Bound\-Val\-Probs} class, and overload the \href{../html/classinit_bound_val_probs.html#ac8f2c3e2a1040c70b709900dc3dfdaea}{\tt get\-\_\-residual()} function\-: 
\begin{DoxyCode}
\textcolor{keyword}{template} <\textcolor{keywordtype}{int} dim>
\textcolor{keyword}{class }CahnHilliard: \textcolor{keyword}{public} initBoundValProbs<dim>
\{
    \textcolor{keyword}{public}:
        CahnHilliard(std::vector<std::vector<std::string> > \_primary\_variables, std::vector<
      std::vector<int> > \_FE\_support, ParameterHandler& \_params);
        \textcolor{comment}{//this is a overloaded function }
        \textcolor{keywordtype}{void} get_residual(\textcolor{keyword}{const} \textcolor{keyword}{typename} hp::DoFHandler<dim>::active\_cell\_iterator &cell, \textcolor{keyword}{const} 
      FEValues<dim>& fe\_values, Table<1, Sacado::Fad::DFad<double> >& R, Table<1, Sacado::Fad::DFad<double>>& ULocal, 
      Table<1, double >& ULocalConv);
        ParameterHandler* params;       
\};
\end{DoxyCode}
 In the overloaded {\bfseries get\-\_\-residual} function, we define the residual for our problem. As our equations are one standrad diffusion equation and Possion equation, we can simiply use the pre-\/defined model\-: 
\begin{DoxyCode}
this->ResidualEq.residualForDiffusionEq(fe\_values, c\_dof, R, c\_1, c\_1\_conv, j\_c\_1);
this->ResidualEq.residualForPoissonEq(fe\_values, mu\_dof, R, kappa\_c\_1\_grad, rhs\_mu);
\end{DoxyCode}
 Though before we call these two functions, we need the flux and reactions terms, we need to first evaluate the values of the primary fields and their spatial gradients. We also need to evaluate the value of primary fields at previous time step for the Backward Euler time scheme\-: 
\begin{DoxyCode}
dealii::Table<1,double>  c\_1\_conv(n\_q\_points);
dealii::Table<1,Sacado::Fad::DFad<double> > c\_1(n\_q\_points), mu(n\_q\_points);
dealii::Table<2,Sacado::Fad::DFad<double> >  c\_1\_grad(n\_q\_points, dim), mu\_grad(n\_q\_points, dim);

evaluateScalarFunction<double,dim>(fe\_values, c\_dof, ULocalConv, c\_1\_conv);
evaluateScalarFunction<Sacado::Fad::DFad<double>,dim>(fe\_values, c\_dof, ULocal, c\_1);   
evaluateScalarFunctionGradient<Sacado::Fad::DFad<double>,dim>(fe\_values, c\_dof, ULocal, c\_1\_grad);

evaluateScalarFunction<Sacado::Fad::DFad<double>,dim>(fe\_values, mu\_dof, ULocal, mu);   
evaluateScalarFunctionGradient<Sacado::Fad::DFad<double>,dim>(fe\_values, mu\_dof, ULocal, mu\_grad);

\textcolor{comment}{//evaluate diffusion and reaction term}
dealii::Table<1,Sacado::Fad::DFad<double> > rhs\_mu(n\_q\_points);
dealii::Table<2,Sacado::Fad::DFad<double> > j\_c\_1(n\_q\_points, dim), kappa\_c\_1\_grad(n\_q\_points, dim);

j\_c\_1=table\_scaling<Sacado::Fad::DFad<double>, dim>(mu\_grad,-M);\textcolor{comment}{//-D\_1*c\_1\_grad}
kappa\_c\_1\_grad=table\_scaling<Sacado::Fad::DFad<double>, dim>(c\_1\_grad,kappa);

\textcolor{keywordflow}{for}(\textcolor{keywordtype}{unsigned} \textcolor{keywordtype}{int} q=0; q<n\_q\_points;q++) rhs\_mu[q]=2*omega*(c\_1[q]-c\_alpha)*(c\_1[q]-c\_beta)*(2*c\_1[q]-
      c\_alpha-c\_beta)-mu[q];
\end{DoxyCode}


The last thing we need to define is the initial condition, we can simpily overload the \href{../html/class_initial_conditions.html#aa10cfdd7350c3810a8deab707f397657}{\tt vector\-\_\-value()} function of the \href{../html/class_initial_conditions.html}{\tt Initial\-Conditions } class, 
\begin{DoxyCode}
\textcolor{keywordtype}{void} InitialConditions<dim>::vector_value (\textcolor{keyword}{const} Point<dim>   &p, Vector<double>   &values)\textcolor{keyword}{ const}\{
  Assert (values.size() == 2, ExcDimensionMismatch (values.size(), 2));
  values(1) = 0;    
 values(0)= 0.5 + 0.04*(static\_cast <\textcolor{keywordtype}{double}> (rand())/(static\_cast <double>(RAND\_MAX))-0.5);
\}
\end{DoxyCode}
 \hypertarget{growth_results}{}\section{Results}\label{growth_results}
 

The results are generated using paramters shown below. The complete implementaion can be found at \href{https://github.com/mechanoChem/mechanoChemFEM/tree/example/Example2_CahnHilliard}{\tt Github}.


\begin{DoxyCode}
\textcolor{preprocessor}{#parameters file}
\textcolor{preprocessor}{}
subsection Problem
set print\_parameter = \textcolor{keyword}{true}

set dt = 5
set totalTime = 250
set current\_increment = 0
set off\_output\_index=0
set current\_time = 0
set resuming\_from\_snapshot = \textcolor{keyword}{false}

set output\_directory = output/
set snapshot\_directory = snapshot/

\textcolor{preprocessor}{#FEM}
\textcolor{preprocessor}{}set volume\_quadrature = 3 
set face\_quadrature = 2 

end

subsection Geometry
set X\_0 = 0
set Y\_0 = 0
set Z\_0 = 0
set X\_end = 1 
set Y\_end = 1
set Z\_end = 2.0 #no need to 2D

set element\_div\_x=50
set element\_div\_y=50
set element\_div\_z=5 #no need to 2D
end

subsection Concentration
set omega = 0.25
set c\_alpha = 0.2
set c\_beta = 0.8
set kappa = 0.002
set M=0.1
end
                        
\textcolor{preprocessor}{#}
\textcolor{preprocessor}{}\textcolor{preprocessor}{# parameters reserved for deal.ii first level code:}
\textcolor{preprocessor}{}\textcolor{preprocessor}{#nonLinear\_method : classicNewton}
\textcolor{preprocessor}{}\textcolor{preprocessor}{#solver\_method (direct) : PETScsuperLU, PETScMUMPS}
\textcolor{preprocessor}{}\textcolor{preprocessor}{#solver\_method (iterative) : PETScGMRES PETScBoomerAMG}
\textcolor{preprocessor}{}\textcolor{preprocessor}{#relative\_norm\_tolerance, absolute\_norm\_tolerance, max\_iterations}
\textcolor{preprocessor}{}\textcolor{preprocessor}{#}
\textcolor{preprocessor}{}subsection Nonlinear\_solver
        set nonLinear\_method = classicNewton
        set relative\_norm\_tolerance = 1.0e-12
        set absolute\_norm\_tolerance = 1.0e-12
        set max\_iterations = 10
end
                        
subsection Linear\_solver
        set solver\_method = PETScsuperLU
        set system\_matrix\_symmetricFlag = \textcolor{keyword}{false} # \textcolor{keywordflow}{default} is \textcolor{keyword}{false}
end
\end{DoxyCode}
 