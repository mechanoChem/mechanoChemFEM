\hypertarget{_intercalation_Introduction}{}\section{Introduction}\label{_intercalation_Introduction}
A battery cell usually consists of porous, positive and negative electrodes, a separator and a current collector. This example present coupled continuum formulation for the electrostatic, chemical, thermal and mechanical processes in battery materials. The treatment applies on the macroscopic scale, at which electrodes can be modelled as porous materials made up of active particles held together by binders and perfused by the electrolyte.

This code is used to generate results for paper\-: {\bfseries Intercalation driven porosity effects on the electro-\/chemo-\/thermo-\/mechanical response in continuum models for battery material electrodes}, Z. Wang, J. Siegel, K. Garikipati, Journal of the Electrochemical Society, Vol. 164\-: A2199-\/\-A2212, 2017, doi\-:10.\-1149/2.0081712jes  \hypertarget{battery_particle_section1}{}\subsection{The multi-\/physics particle scale model}\label{battery_particle_section1}
\hypertarget{battery_particle_subsub1}{}\subsubsection{Electro-\/chemo-\/thermal equations}\label{battery_particle_subsub1}
{\bfseries The electrochemical equations for finitely deforming electrodes}\par
 The ordinary differential equation for mass balance of lithium\-: \[ \frac{\partial}{\partial t}({\epsilon_\text{s}C_\text{Li}})+\epsilon_\text{s}{a_\text{p}}j_n=0 \] where $C_\text{Li}$ is the concentration of lithium in the deformed configuration, $\epsilon_\text{s}$ is the volume fraction of solid particles, and $a_\text{p}$ is related to the inverse radius. For spherical particles of radius $R_\text{p}$, it is defined as $a_\text{p} = 3/R_\text{p}$. Finally, $j_n$ is the normal flux of lithium on the particle's surface.

The partial differential equation for mass balance of lithium ions, $C_{\text{Li}^+}$\-: \[ \frac{\partial }{\partial t}(\epsilon_\text{l}C_{\text{Li}^+})=\nabla\cdot (\epsilon_\text{l}D_\text{eff}\nabla C_{\text{Li}^+})+(1-t^0_+)\epsilon_\text{s}{a_\text{p}}j_n. \] where $\epsilon_\text{l}$ is the volume fraction of pores in the electrode, $t^0_+$ is the lithium ion transference number, and \$\-D\-\_\-\{eff\}\$ is the effective diffusivity.

The equations for the electric fields are, \[ \nabla\cdot\left(\gamma_\text{eff}(-\nabla\phi_\text{E})+\gamma_\text{eff}\frac{2R\theta}{F}(1-t^{0}_{+})\nabla\ln C_{2}\right)=a_\text{p}Fj_{n} \] \[ \nabla\cdot\left(\sigma_\text{eff}(-\nabla\phi_\text{S})\right)=-a_\text{p}Fj_{n} \] where $\phi_\text{E}$ and $\phi_\text{S}$ are, respectively, the electric potential fields in the electrolyte and solid, $\gamma_\text{eff}$ and $\sigma_\text{eff}$ are the corresponding effective conductivities which depend on the porosity, $R$ is the universal gas constant and $\theta$ is the temperature.

The electrochemical equations are completed with specification of the Butler-\/\-Volmer model for the flux of lithium, $j_n$\-: \[ j_n=j_0\left(\text{exp}\left(\frac{\alpha_aF}{R\theta}(\phi_\text{S}-\phi_\text{E}-U)\right)-\text{exp}\left(-\frac{\alpha_aF}{R\theta}(\phi_\text{S}-\phi_\text{E}-U)\right)\right)\\ j_0=k_0(C_2)^{\alpha_a} \frac{(C_1^{\text{max}}-C_{1_\text{Li}})^{\alpha_a}}{(C_1^{\text{max}})^{\alpha_a}} \frac{(C_{1_\text{Li}})^{\alpha_c}}{{(C_1^\text{max}})^{\alpha_c}} \] where $\alpha_a, \alpha_c$ are transfer coefficients, $k_0$ is a kinetic rate constant,and $C^{\text{max}}_\text{Li}$ is the maximum concentration of lithium that the particle can contain. The open circuit potential $U$ can be written as a fit depend on $C_\text{Li}$.

{\bfseries The standard thermal equations}\par
 Heat generation and transport are governed by the heat equation, which is derived from the first law of thermodynamics. For the temperature \$\$, we have the standard form of the heat equation in the electrodes\-: \[ \rho C_p\frac{\text{d}\theta}{\text{d}t}=\lambda \nabla^2\theta+Q_{\text{rxn}}+Q_{\text{rev}}+Q_{\text{ohm}} \label{eq:ThermalElectrode} \] where f $\rho$ is the mass density of the electrode, $C_p$ is specific heat and $\lambda$ is the thermal conductivity. In the separator, we have\-: \[ \rho C_p\frac{\text{d}\theta}{\text{d}t}=\lambda \nabla^2\theta+Q_{\text{ohm}}. \] The heat generation terms are\-: \[ Q_{\text{rxn}}=Faj_n(\phi_\text{S}-\phi_\text{E}-U) \quad \text{irreversible entropic heat}\\ Q_{\text{rev}}=Faj_n\theta\frac{\partial U}{\partial \theta} \quad\text{reversible entropic heat}\\ Q_{\text{ohm}}=-\boldsymbol{i}_1\cdot\nabla\phi_\text{S}-\boldsymbol{i}_2\cdot\nabla\phi_\text{E} \quad\text{Joule heating in electrode}\\ Q_{\text{ohm}}=-\boldsymbol{i}_2\cdot\nabla\phi_\text{E} \quad\text{Joule heating in separator} \]\hypertarget{battery_electrode_scale_subsub2}{}\subsubsection{Finite strain mechanics and the evolving porosity model}\label{battery_electrode_scale_subsub2}
Lithium intercalation/de-\/intercalation induces particle swelling/contraction which manifests itself as electrode deformation at the macro-\/scale. Additionally, the particle and separator undergo thermal expansion.\{\{blue\}The electrolyte is assumed not to undergo thermal expansion. Therefore the decomposition of the deformation into elastic, swelling and thermal contributions does not apply to it.\} The kinematics of finite strain leads to the following decomposition\-: \[ \boldsymbol{F}=\boldsymbol{F}^\text{e}\boldsymbol{F}^\text{c}\boldsymbol{F}^{\theta} \] where $\boldsymbol{F} = \boldsymbol{1} + \partial\boldsymbol{u}/\partial\boldsymbol{X}$, is the total deformation gradient tensor averaged over the constituents of the material. It is multiplicatively decomposed into $\boldsymbol{F}^\text{e}$, $\boldsymbol{F}^\text{c}$ and $\boldsymbol{F}^{\theta}$, which are, respectively, its elastic, chemical (induced by lithium intercalation) and thermal components. In the absence of a body force the strong form of the mechanics problem in the current configuration is \[ \nabla\cdot\boldsymbol{T} = \boldsymbol{0} \\ \boldsymbol{T}= \frac{1}{\det{\boldsymbol{F}^\text{e}}}\frac{\partial W}{\partial \boldsymbol{F}^\text{e}}\boldsymbol{F}^{\text{e}^\text{T}} \] where \$\$ is the Cauchy stress tensor and \$\-W\$ is the strain energy density function.

we assume that although the particles undergo isotropic swelling in the electrode due to intercalation of lithium, there is unidirectional swelling along the normal to the separator. This assumption models the non-\/slip boundary condition that would be applied at the current collector-\/electrode interface, and which would provide a strong constraint against macroscopic expansion of the electrode in the e1 and e3 directions. Since we do not directly model the current collector, this assumption represents its mechanical effect Accordingly, we have \[ F^\text{c}_{iJ}=\beta\delta_{2i}\delta_{2J}+\delta_{iJ} \] Thermal expansion is modelled as isotropic \[ F^{\theta}_{iJ}=(1+\beta^{\theta})^{1/3}\delta_{iJ} \] Based on mixture theory (Z. Wang et.\-al.), the volume fraction of particle in electrode can be modeled by \[ \epsilon_\text{s}=\frac{\left( \frac{\kappa(\frac{\det\boldsymbol{F}}{(1+\beta)(1+\beta^\theta)}-1)-3P_\text{l}-3P_\text{b}}{\kappa_\text{s}} +1\right)(1+\beta_\text{s})(1+\beta_\text{s}^\theta)}{\det\boldsymbol{F}}\epsilon_{\text{s}_0} \] Assuming the binder to deform at constant volume during charging and discharging such that \$\{det\}\{F\}\-\_\-\{b\}=1\$, we have \[ \epsilon_\text{b}=\frac{1}{\det \boldsymbol{F}}\epsilon_{\text{b}_0}\\ \epsilon_\text{l}=1-\epsilon_{\text{s}}-\epsilon_\text{b} \] \hypertarget{battery_electrode_scale_sub2}{}\subsection{Boundary condition}\label{battery_electrode_scale_sub2}
 \hypertarget{battery_particle_Implementation}{}\section{Implementation}\label{battery_particle_Implementation}
The coding flow is very similar to \hyperlink{_intercalation}{Example 1 \-: Intercalation with finte strain mechanics}. We discuss the difference here.\hypertarget{battery_electrode_scale_sub1}{}\subsection{Read parameters.\-prm before defining primary variables over different domains}\label{battery_electrode_scale_sub1}
Here, we want to read the paramters from file at the very beginning of main.\-cc instead of at run.\-cc. This is can be easily done at main.\-cc 
\begin{DoxyCode}
ParameterHandler params;
initBoundValProbs<DIMS> problem(params);
ElectricChemo<Sacado::Fad::DFad<double>,DIMS> \_electricChemoFormula(params);
problem.electricChemoFormula=&\_electricChemoFormula;
problem.declare\_parameters();
params.read\_input (\textcolor{stringliteral}{"parameters.prm"});
\end{DoxyCode}
 Then we can read the domain id, and Define primary variables over different domains just like Example 1 
\begin{DoxyCode}
params.enter\_subsection(\textcolor{stringliteral}{"Problem"}); 
\textcolor{keywordtype}{int} separator\_fe=params.get\_integer(\textcolor{stringliteral}{"separator\_fe"});
\textcolor{keywordtype}{int} electrode\_fe=params.get\_integer(\textcolor{stringliteral}{"electrode\_fe"});
params.leave\_subsection();

\textcolor{comment}{//main fields }
std::vector<std::vector<std::string> > primary\_variables(6);\textcolor{comment}{//u; c\_li\_plus;phi\_e;c\_li;phi\_s;T   }
primary\_variables[0].push\_back(\textcolor{stringliteral}{"u"}); primary\_variables[0].push\_back(\textcolor{stringliteral}{"component\_is\_vector"});
primary\_variables[1].push\_back(\textcolor{stringliteral}{"C\_li\_plus"}); primary\_variables[1].push\_back(\textcolor{stringliteral}{"component\_is\_scalar"});
primary\_variables[2].push\_back(\textcolor{stringliteral}{"phi\_e"}); primary\_variables[2].push\_back(\textcolor{stringliteral}{"component\_is\_scalar"});
primary\_variables[3].push\_back(\textcolor{stringliteral}{"C\_li"}); primary\_variables[3].push\_back(\textcolor{stringliteral}{"component\_is\_scalar"});
primary\_variables[4].push\_back(\textcolor{stringliteral}{"phi\_s"}); primary\_variables[4].push\_back(\textcolor{stringliteral}{"component\_is\_scalar"});
primary\_variables[5].push\_back(\textcolor{stringliteral}{"T"}); primary\_variables[5].push\_back(\textcolor{stringliteral}{"component\_is\_scalar"});

\textcolor{comment}{//active material, separator, currentCollector, pure solid}
\textcolor{keywordtype}{int} number\_domain=2;
std::vector<std::vector<int> > FE\_support(number\_domain);\textcolor{comment}{// store order of polynomial basis functions, 0
       means FE\_Nothing   }
\textcolor{comment}{//electrode domain}
FE\_support[separator\_fe].push\_back(1);\textcolor{comment}{//u}
FE\_support[separator\_fe].push\_back(1);\textcolor{comment}{//C\_li\_plus}
FE\_support[separator\_fe].push\_back(1);\textcolor{comment}{//phi\_e}
FE\_support[separator\_fe].push\_back(0);\textcolor{comment}{//C\_li}
FE\_support[separator\_fe].push\_back(0);\textcolor{comment}{//phi\_s}
FE\_support[separator\_fe].push\_back(1);\textcolor{comment}{//T}

\textcolor{comment}{//separator}
FE\_support[electrode\_fe].push\_back(1);\textcolor{comment}{//u}
FE\_support[electrode\_fe].push\_back(1);\textcolor{comment}{//C\_li\_plus}
FE\_support[electrode\_fe].push\_back(1);\textcolor{comment}{//phi\_e}
FE\_support[electrode\_fe].push\_back(1);\textcolor{comment}{//C\_li}
FE\_support[electrode\_fe].push\_back(1);\textcolor{comment}{//phi\_s}
FE\_support[electrode\_fe].push\_back(1);\textcolor{comment}{//T}
\end{DoxyCode}
\hypertarget{battery_electrode_scale_Generate}{}\subsection{Generate mesh internally}\label{battery_electrode_scale_Generate}
In Example 1, mesh is imported from existing mesh. Here we use deal.\-ii {\bfseries Grid\-Generator} to generate mesh 
\begin{DoxyCode}
\textcolor{keywordflow}{for} (\textcolor{keywordtype}{unsigned} \textcolor{keywordtype}{int} j = 0; j < element\_div\_x; ++j) step\_sizes[0].push\_back((X\_end-X\_0)/element\_div\_x); 
\textcolor{keywordflow}{for} (\textcolor{keywordtype}{unsigned} \textcolor{keywordtype}{int} j = 0; j < element\_div\_y; ++j) step\_sizes[1].push\_back((Y\_end-Y\_0)/element\_div\_y);
\textcolor{keywordflow}{if}(dim==3)\textcolor{keywordflow}{for} (\textcolor{keywordtype}{unsigned} \textcolor{keywordtype}{int} j = 0; j < element\_div\_z; ++j) step\_sizes[2].push\_back((Z\_end-Z\_0)/
      element\_div\_z);

\textcolor{keywordflow}{if}(dim==2) GridGenerator::subdivided\_hyper\_rectangle (this->triangulation, step\_sizes, Point<dim>(X\_0,Y\_0),
       Point<dim>(X\_end,Y\_end), colorize);
\textcolor{keywordflow}{else} GridGenerator::subdivided\_hyper\_rectangle (this->triangulation, step\_sizes, Point<dim>(X\_0,Y\_0,Z\_0), 
      Point<dim>(X\_end,Y\_end,Z\_end), colorize);
\end{DoxyCode}
\hypertarget{battery_electrode_scale_Finite}{}\subsection{Finite strain mechanics using Saint Venant Kirchhoff model}\label{battery_electrode_scale_Finite}
After we have elastic deformation gradient tensor {\bfseries Fe}, we need to choose the constitutive model for mechanics 
\begin{DoxyCode}
Residual<vectorType,dim>::evaluateStrain(Fe, E, defMap, infinitesimal\_strain\_indicator);
Residual<vectorType,dim>::evaluateSaint_Venant_KirchhoffStress(P\_stress,Fe, E);
Residual<vectorType,dim>::residualForMechanics(fe\_values, u\_dof, R, P\_stress);  
\end{DoxyCode}
\hypertarget{battery_electrode_scale_neumann}{}\subsection{Apply neumann boundary condition}\label{battery_electrode_scale_neumann}
In this example we need to apply neumann boundary condition for heat dissipation and current. We need to first find the corresponding surface, and use function {\bfseries residual\-For\-Neumman\-B\-C}. 
\begin{DoxyCode}
\textcolor{keywordflow}{for} (\textcolor{keywordtype}{unsigned} \textcolor{keywordtype}{int} faceID=0; faceID<2*dim; faceID++)\{
    \textcolor{keywordflow}{if}(cell->face(faceID)->at\_boundary())\{
        FEFaceValues<dim>* fe\_face\_values;
        \textcolor{keywordflow}{if}(domain==1 or domain==-1) fe\_face\_values=&electrode\_fe\_face\_values;
        \textcolor{keywordflow}{else} \textcolor{keywordflow}{if}(domain==0) fe\_face\_values=&separator\_fe\_face\_values;
        fe\_face\_values->reinit (cell, faceID);
        \textcolor{keyword}{const} \textcolor{keywordtype}{unsigned} \textcolor{keywordtype}{int} n\_face\_quadrature\_points = fe\_face\_values->n\_quadrature\_points;
        dealii::Table<1,Sacado::Fad::DFad<double> > T\_face(n\_face\_quadrature\_points),heat\_transfer(
      n\_face\_quadrature\_points);
        evaluateScalarFunction(fe\_values, *fe\_face\_values, T\_dof, ULocal, T\_face);
        \textcolor{keywordflow}{for}(\textcolor{keywordtype}{unsigned} \textcolor{keywordtype}{int} q=0;q<n\_face\_quadrature\_points;q++) heat\_transfer[q]=h*(T\_face[q]-T\_ini);
        Residual<vectorType,dim>::residualForNeummanBC(fe\_values, *fe\_face\_values, T\_dof, R, heat\_transfer)
      ;
        \textcolor{keywordflow}{if}(cell->face(faceID)->boundary\_id()==dim*2 )\{
            \textcolor{keywordtype}{double} current;
            \textcolor{keywordflow}{if}(cell\_center[1]<=electrode\_Y1) current=current\_IpA;
            \textcolor{keywordflow}{else} current=-current\_IpA;
            Residual<vectorType,dim>::residualForNeummanBC(fe\_values, *fe\_face\_values, phi\_s\_dof, R, 
      current);
        \}
    \}
\}   
\end{DoxyCode}
 \hypertarget{battery_electrode_scale_Scalling}{}\subsection{Residual scalling}\label{battery_electrode_scale_Scalling}
In battery modelling, the value of parameters may differ a lot, and scalling may be necessary before solving. And this is can be easilly done using function {\bfseries scalling}. 
\begin{DoxyCode}
ResidualEq->scalling(fe\_values,c\_li\_plus\_dof,R,1e-3);
ResidualEq->scalling(fe\_values,phi\_e\_dof,R,1e-3);
ResidualEq->scalling(fe\_values,c\_li\_dof,R,1e-5);
ResidualEq->scalling(fe\_values,phi\_s\_dof,R,1e-6);
ResidualEq->scalling(fe\_values,T\_dof,R,1e-5);
\end{DoxyCode}
\hypertarget{battery_electrode_scale_computed}{}\subsection{Output extral field computed from primary variabls}\label{battery_electrode_scale_computed}
During battery simulation, reaction rate is a important intermidate results and we want to output it along with the primary variables. To do so we first need to implement a class derived from {\bfseries computed\-Field$<$dim$>$}. 
\begin{DoxyCode}
\textcolor{preprocessor}{#include <deal.II/base/parameter\_handler.h>}
\textcolor{preprocessor}{#include <supplementary/computedField.h>}
\textcolor{preprocessor}{#include "electricChemo.h"}
\(\backslash\)code\{.cpp\}
\textcolor{keyword}{template} <\textcolor{keywordtype}{int} dim>
\textcolor{keyword}{class }nodalField : \textcolor{keyword}{public} computedField<dim>
\{
\textcolor{keyword}{public}:
    nodalField(dealii::ParameterHandler& \_params);
    ~nodalField();
    
    dealii::ParameterHandler* params;
    \textcolor{keywordtype}{void} compute\_derived\_quantities\_vector(\textcolor{keyword}{const} std::vector<Vector<double> > &uh,
                           \textcolor{keyword}{const} std::vector<std::vector<Tensor<1,dim> > > &duh,
                           \textcolor{keyword}{const} std::vector<std::vector<Tensor<2,dim> > > &dduh,
                           \textcolor{keyword}{const} std::vector<Point<dim> >                  &normals,
                           \textcolor{keyword}{const} std::vector<Point<dim> >                  &evaluation\_points,
                           std::vector<Vector<double> >                    &computed\_quantities) \textcolor{keyword}{const};
                                 
    
                                 \textcolor{comment}{//std::vector<std::string> get\_names () const;}
                                 \textcolor{comment}{//std::vector<DataComponentInterpretation::DataComponentInterpretation>
       get\_data\_component\_interpretation () const;}
                                 \textcolor{comment}{//virtual UpdateFlags get\_needed\_update\_flags () const;}
                                 \textcolor{comment}{//void setupComputedField(std::vector<std::vector<std::string> >
       \_primary\_variables);}
                                                             
    std::vector<unsigned int > primary\_variables\_dof;
    \textcolor{comment}{//std:}
\end{DoxyCode}
 And overload function {\bfseries compute\-\_\-derived\-\_\-quantities\-\_\-vector} for the new field\-: 
\begin{DoxyCode}
\textcolor{keyword}{template} <\textcolor{keywordtype}{int} dim>
\textcolor{keywordtype}{void} nodalField<dim>::compute\_derived\_quantities\_vector(\textcolor{keyword}{const} std::vector<Vector<double> > &uh,
                           \textcolor{keyword}{const} std::vector<std::vector<Tensor<1,dim> > > &duh,
                           \textcolor{keyword}{const} std::vector<std::vector<Tensor<2,dim> > > &dduh,
                           \textcolor{keyword}{const} std::vector<Point<dim> >                  &normals,
                           \textcolor{keyword}{const} std::vector<Point<dim> >                  &evaluation\_points,
                           std::vector<Vector<double> >                    &computed\_quantities)\textcolor{keyword}{ const}
\textcolor{keyword}{}\{
    ElectricChemo<double,dim> electricChemoFormula;
    electricChemoFormula.params=params;
    electricChemoFormula.setParametersFromHandler();

    params->enter\_subsection(\textcolor{stringliteral}{"Geometry"});
    \textcolor{keywordtype}{double} electrode\_Y1=params->get\_double(\textcolor{stringliteral}{"electrode\_Y1"});
    \textcolor{keywordtype}{double} electrode\_Y2=params->get\_double(\textcolor{stringliteral}{"electrode\_Y2"});
  params->leave\_subsection();
    
    \textcolor{keywordtype}{int} u\_dof=primary\_variables\_dof[0];
    \textcolor{keywordtype}{int} c\_li\_plus\_dof=primary\_variables\_dof[1];
    \textcolor{keywordtype}{int} phi\_e\_dof=primary\_variables\_dof[2];
    \textcolor{keywordtype}{int} c\_li\_dof=primary\_variables\_dof[3];
    \textcolor{keywordtype}{int} phi\_s\_dof=primary\_variables\_dof[4];
    \textcolor{keywordtype}{int} T\_dof=primary\_variables\_dof[5];
    
    \textcolor{keyword}{const} \textcolor{keywordtype}{unsigned} \textcolor{keywordtype}{int} dof\_per\_node = uh.size();
    \textcolor{keywordflow}{for} (\textcolor{keywordtype}{unsigned} \textcolor{keywordtype}{int} q=0; q<dof\_per\_node; ++q)\{
        computed\_quantities[q][0]=0;
        
        \textcolor{keywordtype}{double} c\_li\_plus, phi\_e, c\_li, phi\_s, T;
        c\_li\_plus=uh[q][c\_li\_plus\_dof];
        phi\_e=uh[q][phi\_e\_dof];
        c\_li=uh[q][c\_li\_dof];
        phi\_s=uh[q][phi\_s\_dof];
        T=uh[q][T\_dof];
        \textcolor{keywordtype}{int} domain;
        
        \textcolor{keywordflow}{if}(evaluation\_points[q][1]<=electrode\_Y1) domain=-1;
        \textcolor{keywordflow}{else} \textcolor{keywordflow}{if}(evaluation\_points[q][1]>=electrode\_Y2)domain=1;
        \textcolor{keywordflow}{else} domain=0;
        computed\_quantities[q][0]=electricChemoFormula.formula\_jn(T, c\_li, c\_li\_plus, phi\_s, phi\_e, domain)
      ;        
    \}   
\}
\end{DoxyCode}


Then before write K\-T\-K file, we need to pass the class to {\bfseries F\-E\-Mdata$<$dim,\-P\-E\-T\-Sc\-Wrappers\-::\-M\-P\-I\-::\-Vector$>$} 
\begin{DoxyCode}
    nodalField<dim> computedNodalField(*params);
    std::vector<std::vector<std::string> > computed\_primary\_variables=\{\{\textcolor{stringliteral}{"jn"}, \textcolor{stringliteral}{"component\_is\_scalar"}\}\};
    \textcolor{comment}{//computed\_primary\_variables[0].push\_back("jn");
       computed\_primary\_variables[0].push\_back("component\_is\_scalar");}
    computedNodalField.setupComputedField(computed\_primary\_variables);
    computedNodalField.primary\_variables\_dof=primary\_variables\_dof;

...
    FEMdata<dim,vectorType>::data_out.add\_data\_vector (localized\_U, computedNodalField);
    std::string output\_path = output\_directory+\textcolor{stringliteral}{"output-"}+std::to\_string(current\_increment)+\textcolor{stringliteral}{".vtk"};
    FEMdata<dim,vectorType>::write_vtk(solution\_prev, output\_path); 
\end{DoxyCode}
\hypertarget{battery_electrode_scale_Pa}{}\subsubsection{Parameterhandler is used to manage all parameters as Example 1}\label{battery_electrode_scale_Pa}
\hypertarget{_intercalation_results}{}\section{Results}\label{_intercalation_results}
We present few results here by using the following Parameter file\-: 
\begin{DoxyCode}
\textcolor{preprocessor}{#parameters file}
\textcolor{preprocessor}{}\textcolor{preprocessor}{#1. Thermal modeling of cylindrical lithium ion battery during discharge cycle, Dong Hyup Jeon ⇑,1, Seung
       Man Baek,2011,Energy Conversion and Management LiC6}
\textcolor{preprocessor}{}\textcolor{preprocessor}{# 2. Numerical study on lithium titanate battery thermal response under adiabatic condition,Qiujuan Sun a,
       Qingsong Wanga 2015, Energy Conversion and Management}
\textcolor{preprocessor}{}\textcolor{preprocessor}{#3.white Theoretical Analysis of Stresses in a Lithium Ion Cell    }
\textcolor{preprocessor}{}\textcolor{preprocessor}{#4 A Computational Model of the Mechanical Behavior within Reconstructed LixCoO2 Li-ion Battery Cathode
       Particles, Veruska Malavé, J.R. Berger, EA}
\textcolor{preprocessor}{}\textcolor{preprocessor}{#5. A pseudo three-dimensional electrochemicalethermal model of a prismatic LiFePO4 battery during
       discharge process}
\textcolor{preprocessor}{}\textcolor{preprocessor}{#6. solid diffusion Single-Particle Model for a Lithium-Ion Cell: Thermal Behavior, Meng Guo,a Godfrey
       Sikha,b,* and Ralph E. Whitea}
\textcolor{preprocessor}{}\textcolor{preprocessor}{#global parameters}
\textcolor{preprocessor}{}

\textcolor{preprocessor}{#declare problem setting}
\textcolor{preprocessor}{}subsection Problem
set print\_parameter = \textcolor{keyword}{true}
set dt = 10
set totalTime = 370
set step\_load = \textcolor{keyword}{false}

set first\_domain\_id = 0
set electrode\_id=0
set separator\_id=1

set electrode\_fe=0
set separator\_fe=1

\textcolor{preprocessor}{#directory}
\textcolor{preprocessor}{}set output\_directory = output/
set snapshot\_directory = snapshot/ 
\textcolor{preprocessor}{#FEM}
\textcolor{preprocessor}{}set volume\_quadrature = 4 
set face\_quadrature = 3 
\textcolor{preprocessor}{#applied current}
\textcolor{preprocessor}{}set IpA = -100
end

\textcolor{preprocessor}{# some useful geometry information beforehand}
\textcolor{preprocessor}{}subsection Geometry
set X\_0 = 0
set Y\_0 = 0
set Z\_0 = 0
set X\_end = 120
set Y\_end = 120
set Z\_end = 85

set electrode\_Y1 = 60
set electrode\_Y2 = 80

set element\_div\_x = 1
set element\_div\_y = 120
set element\_div\_z = 1
end

\textcolor{preprocessor}{# initial condition}
\textcolor{preprocessor}{}subsection Initial condition
set c\_li\_max\_neg = 28.7e-3
set c\_li\_max\_pos = 37.5e-3
set c\_li\_100\_neg = 0.915
set c\_li\_100\_pos = 0.022
set c\_li\_0\_neg = 0.02
set c\_li\_0\_pos = 0.98
set c\_li\_plus\_ini = 1.0e-3
set T\_0 = 298
end

\textcolor{preprocessor}{# parameter for elastiticy equations}
\textcolor{preprocessor}{}subsection Elasticity
set youngsModulus\_neg = 12e3 #cite 3 in porosity paper
set youngsModulus\_pos = 370e3 #cite4
set youngsModulus\_sep = 0.5e3 #cite 3 in porosity paper

set nu\_sep = 0.35 #cite 3 in porosity paper
set nu\_neg = 0.3 #cite 3 in porosity paper
set nu\_pos = 0.2 #cite4 

set kappa\_sep = 0.42e-3
set kappa\_neg = 4.94e-3
set kappa\_pos = 7.4e-3
set kappa\_s = 25e-3
set pb = 0
set pl = 0
set omega\_s = 3.5 #not used but 3.5 is from linear coff in porosity paper
\textcolor{preprocessor}{#following from cite 3 in porosity paper}
\textcolor{preprocessor}{}set omega\_neg = 9.615e-6 
set omega\_pos = 6.025e-6
set omega\_sep = 82.46e-5

end

\textcolor{preprocessor}{# parameter for electro-chemo equations}
\textcolor{preprocessor}{}subsection ElectroChemo #cite6
set sigma\_neg = 1.5e8
set sigma\_pos = 0.5e8

set t\_0 = 0.2
set D\_li\_neg = 5e-1 #cite 40 in porosity paper in code use expression cite 3 
set D\_li\_pos = 1.0e-1 #cite3

set eps\_s\_0\_neg = 0.53
set eps\_s\_0\_pos = 0.5
set eps\_s\_0\_sep = 0.35

set eps\_l\_0\_neg = 0.32
set eps\_l\_0\_pos = 0.35
set eps\_l\_0\_sep = 0.65

set eps\_b\_0\_neg = 0.15
set eps\_b\_0\_pos = 0.15
set eps\_b\_0\_sep = 0

set R\_s\_0\_neg = 8.0
set R\_s\_0\_pos = 6.0
set R\_s\_0\_sep = 0.0

# parameter \textcolor{keywordflow}{for} Butler-Volmer equations at ElectricChemo \textcolor{keyword}{class}
set F = 96485.3329
set Rr = 8.3144598
set alpha\_a = 0.5
set alpha\_c = 0.5
set k\_neg = 0.8 #to match porosity paper
set k\_pos = 0.8 #to match porosity paper
set c\_max\_neg = 28.7e-3
set c\_max\_pos = 37.5e-3

end

\textcolor{preprocessor}{# parameter for thermal equations}
\textcolor{preprocessor}{}subsection Thermal
set lambda\_neg = 1.04e6 #cite1
set lambda\_pos = 5e6  #cite1
set lambda\_sep = 0.33e6  #cite1


set density\_neg = 2.5e-15  #cite1
set density\_pos = 2.5e-15  #cite1
set density\_sep = 1.1e-15  #cite1


set Cp\_s\_neg = 700e12  #cite1
set Cp\_s\_pos = 700e12 #cite1
set Cp\_sep = 700e12  #cite1


set h = 5 #Heat transfer coefficient
end
            
                    
\textcolor{preprocessor}{#==============================================================================}
\textcolor{preprocessor}{}\textcolor{preprocessor}{# parameters reserved for deal.ii first level code:}
\textcolor{preprocessor}{}\textcolor{preprocessor}{#nonLinear\_method : classicNewton}
\textcolor{preprocessor}{}\textcolor{preprocessor}{#solver\_method (direct) : PETScsuperLU, PETScMUMPS}
\textcolor{preprocessor}{}\textcolor{preprocessor}{#solver\_method (iterative) : PETScGMRES PETScBoomerAMG}
\textcolor{preprocessor}{}\textcolor{preprocessor}{#relative\_norm\_tolerance, absolute\_norm\_tolerance, max\_iterations}
\textcolor{preprocessor}{}\textcolor{preprocessor}{#}
\textcolor{preprocessor}{}subsection Nonlinear\_solver
        set nonLinear\_method = classicNewton
        set relative\_norm\_tolerance = 1.0e-10
        set absolute\_norm\_tolerance = 1.0e-7
        set max\_iterations = 8
end
                        
subsection Linear\_solver
        set solver\_method = PETScMUMPS
        set system\_matrix\_symmetricFlag = \textcolor{keyword}{false} # \textcolor{keywordflow}{default} is \textcolor{keyword}{false}
end
\end{DoxyCode}


   \par
 

Many more results can be found at paper\-: {\bfseries Intercalation driven porosity effects on the electro-\/chemo-\/thermo-\/mechanical response in continuum models for battery material electrodes}, Z. Wang, J. Siegel, K. Garikipati, Journal of the Electrochemical Society, Vol. 164\-: A2199-\/\-A2212, 2017, doi\-:10.\-1149/2.0081712jes \hypertarget{_intercalation_com}{}\section{Complete code}\label{_intercalation_com}
The complete implementaion can be found at \href{https://github.com/mechanoChem/mechanoChemFEM/tree/example/Example2%20Battery%20model%20at%20electrode%20scale}{\tt Github}. 