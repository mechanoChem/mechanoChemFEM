\hypertarget{growth_Introduction}{}\section{Introduction}\label{growth_Introduction}
We solve two diffusion reaction equations\-: \[ \frac{\partial C_\text{1}}{\partial t}+\nabla\cdot\boldsymbol{j}_1=r_1 \\ \frac{\partial C_\text{2}}{\partial t}+\nabla\cdot\boldsymbol{j}_2=r_2 \] where $\boldsymbol{j}_1 $ and $\boldsymbol{j}_2 $ are flux terms\-: \[ \boldsymbol{j}_1^{diff}=-M_1\nabla C_\text{1}; \quad \boldsymbol{j}_2^{diff}=-M_2\nabla C_\text{2}\\ \] $r_1$ and $r_2$ are reaction terms, in this example they are \[ r_1= reac_{10}; \quad r_1= reac_{20} \]

Besides chemistry, we also solve elasticity problem at finite strain\-: \[ \nabla\cdot\boldsymbol{T} = \boldsymbol{0}\\ \boldsymbol{T}= \frac{1}{\det{\boldsymbol{F}^{\text{e}}}}\frac{\partial W}{\partial \boldsymbol{F}^{\text{e}}}\boldsymbol{F}^{\text{e}} \] To make it more interesting, we have mechanical deformation due to species intecalation. \[ \boldsymbol{F}=\boldsymbol{F}^{\text{e}}\boldsymbol{F}^{\text{g}}\\ \boldsymbol{F}^{\text{g}}=\left(\frac{C_\text{1}}{C_\text{10}}\right)^{\frac{1}{3}}\mathbb{1} \] For deomstraion we use a very simple mesh with three different domains. We will solve the abolve equations on domain 1 and 2.  \hypertarget{brain_morph_define1}{}\subsection{Define primary variables over different domains}\label{brain_morph_define1}
In deal.\-ii to solve equations in different domains, {\ttfamily Fe\-\_\-\-Nothing} is used and different {\ttfamily Fe\-\_\-\-System} need to be defined. By using deal\-Mutiphysics, it can be easily taken care of by two user defined vector\-: {\ttfamily {\bfseries primary\-\_\-variables}} and {\ttfamily {\bfseries F\-E\-\_\-support}}. 
\begin{DoxyCode}
       std::vector<std::vector<std::string> > primary\_variables(3);        
       primary\_variables[0].push\_back(\textcolor{stringliteral}{"u"}); primary\_variables[0].push\_back(\textcolor{stringliteral}{"component\_is\_vector"});
       primary\_variables[1].push\_back(\textcolor{stringliteral}{"c1"}); primary\_variables[1].push\_back(\textcolor{stringliteral}{"component\_is\_scalar"});
       primary\_variables[2].push\_back(\textcolor{stringliteral}{"c2"}); primary\_variables[2].push\_back(\textcolor{stringliteral}{"component\_is\_scalar"});
       
       \textcolor{keywordtype}{int} number\_domain=3;
       \textcolor{keywordtype}{int} mech\_degree=1;
       \textcolor{keywordtype}{int} diff\_degree=1;
       std::vector<std::vector<int> > FE\_support(number\_domain);\textcolor{comment}{// store order of polynomial basis
       functions, 0 means FE\_Nothing   }
       \textcolor{keywordflow}{for}(\textcolor{keywordtype}{unsigned} \textcolor{keywordtype}{int} i=0;i<2;i++)\{
           FE\_support[i].push\_back(mech\_degree);
           FE\_support[i].push\_back(diff\_degree);
           FE\_support[i].push\_back(diff\_degree);
       \}
       FE\_support[2].push\_back(0);
       FE\_support[2].push\_back(0);
       FE\_support[2].push\_back(0);
*
\end{DoxyCode}
 Then user can use function 
\begin{DoxyCode}
 *  hpFEM<dim>::setup_FeSystem(fe\_system, fe\_collection, q\_collection, primary\_variables\_dof,
      primary\_variables,FE\_support,volume\_quadrature);
* 
\end{DoxyCode}
 to setup {\ttfamily fe\-\_\-system}, {\ttfamily fe\-\_\-collection}, {\ttfamily q\-\_\-collection} with quadarture points \-: {\ttfamily volume\-\_\-quadrature}.\par
 If the mesh is generated outside deal.\-ii (e.\-g. cubit), material I\-D can be pre-\/defined. To setup the {\ttfamily Fe\-\_\-\-System} for each element, we can simily use 
\begin{DoxyCode}
 *  hpFEM<dim>::set_active_fe_indices (FE\_support, hpFEM<dim>::dof_handler);
* 
\end{DoxyCode}
 \hypertarget{brain_morph_outPut}{}\subsection{Output and restart}\label{brain_morph_outPut}
We can use class {\ttfamily F\-E\-Mdata} to easily write output and have capability of re-\/start the code. After initializing the class we can set the output name by 
\begin{DoxyCode}
 *  FEMdata<dim,vectorType>::set_output_name(primary\_variables);
* 
\end{DoxyCode}
 And write vtk file using 
\begin{DoxyCode}
  std::string output\_path = output\_directory+\textcolor{stringliteral}{"output-0.vtk"};
  FEMdata<dim,vectorType>::write_vtk(solution\_0, output\_path);
* 
\end{DoxyCode}
 To use restart we first need to create a snapshot and resume the vector from ths snapshot 
\begin{DoxyCode}
        std::string snapshot\_path = snapshot\_directory+\textcolor{stringliteral}{"snapshot-"}+std::to\_string(current\_increment)+\textcolor{stringliteral}{".txt"}
      ;
        FEMdata<dim,vectorType>::create_vector_snapshot(solution\_old, snapshot\_path);
        FEMdata<dim,vectorType>::resume_vector_from_snapshot(solution\_new, snapshot\_path);
* 
\end{DoxyCode}
 \hypertarget{brain_morph_assemble}{}\subsection{Assemble rsesidual functions of two diffuction reactions equations and elasticity at finite strain}\label{brain_morph_assemble}
In F\-E\-M modeling, we need to provide system\-\_\-matrix (Jacobin matrix) $\frac{\partial \boldsymbol{R}}{\partial \boldsymbol{x}} $, and right hand side vector $-\boldsymbol{R}$. They are usually achieved by assemble {\ttfamily local\-\_\-matrix} and {\ttfamily local\-\_\-rhs} over elements. We first need to overload abstract function {\ttfamily update\-Linear\-System}. In this exmaple we need to solve two diffusion-\/reaction equations and one finite strain mechanics problem. \hypertarget{brain_morph_DRq}{}\subsubsection{diffusion-\/reaction equations}\label{brain_morph_DRq}
For diffusion-\/reaction problems we need to provide the flux and reaction term, and in this example we also have advection and few extra terms for stablization. All the term will be combined into {\ttfamily flux} and {\ttfamily reaction} terms. \[ \boldsymbol{j}_1^{diff}=-M_1\nabla C_\text{1}; \quad \boldsymbol{j}_2^{diff}=-M_2\nabla C_\text{2}\\ \] The following code basically evaluates the terms shown above. 
\begin{DoxyCode}
\textcolor{keywordflow}{for}(\textcolor{keywordtype}{unsigned} \textcolor{keywordtype}{int} q=0; q<n\_q\_points;q++)\{
    reac\_11[q]=0; reac\_12[q]=0; c\_1\_reac[q]=0; 
    reac\_21[q]=0; reac\_22[q]=0; c\_2\_reac[q]=0; 
    velDotGradSpatc\_1[q]=0;
    velDotGradSpatc\_2[q]=0;
    \textcolor{keyword}{const} Point<dim> posR = fe\_values.quadrature\_point(q);
    \textcolor{keywordflow}{for}(\textcolor{keywordtype}{unsigned} \textcolor{keywordtype}{int} i=0; i<dim;i++)\{
        c\_1\_flux[q][i]=-mobility\_c1*c\_1\_grad[q][i];
        c\_2\_flux[q][i]=-mobility\_c2*c\_2\_grad[q][i];
    \}       
    c\_1\_reaction[q]=reac\_10;
    c\_2\_reaction[q]=reac\_20;
\}
*
\end{DoxyCode}
 Also in above code we need values for primary variables and their gradients (at current configuration), we can evaluate them using {\ttfamily evaluation} {\ttfamily functions\-:} 
\begin{DoxyCode}
            evaluateScalarFunction(fe\_values, primary\_variables\_dof[1], ULocalConv, c\_1\_conv);
            evaluateScalarFunction(fe\_values, primary\_variables\_dof[1], U0Local, c\_1\_0);
            evaluateScalarFunction(fe\_values, primary\_variables\_dof[1], ULocal, c\_1);   
            evaluateScalarFunctionGradient(fe\_values, primary\_variables\_dof[1], ULocal, c\_1\_grad,defMap);
            
            evaluateScalarFunction(fe\_values, primary\_variables\_dof[2], ULocalConv, c\_2\_conv);
            evaluateScalarFunction(fe\_values, primary\_variables\_dof[2], U0Local, c\_2\_0);
            evaluateScalarFunction(fe\_values, primary\_variables\_dof[2], ULocal, c\_2);   
            evaluateScalarFunctionGradient(fe\_values, primary\_variables\_dof[2], ULocal, c\_2\_grad,defMap);
*
\end{DoxyCode}
 The weak form of diffution-\/reaction equations can be written as \[ \mathscr{R}=\int_{\Omega_{\text{e}}}w\frac{\partial C}{\partial t}\text{d}v-\int_{\Omega_{\text{e}}} \nabla w \boldsymbol{j} \text{d}v+\int_{s}w\boldsymbol{j}\cdot\boldsymbol{n} \text{d}s=\int_{\Omega_{\text{e}}}w r \text{d}v \] After we have all fluxes and reaction terms we can simply call the corresponding residual function to evaluate the residual for $C_\text{1}$ and $C_\text{2}$ . 
\begin{DoxyCode}
            Residual<vectorType,dim>::residualForDiff_ReacEq(fe\_values, primary\_variables\_dof[1], R, defMap
      , c\_1, c\_1\_conv, c\_1\_flux, c\_1\_reaction);
            Residual<vectorType,dim>::residualForDiff_ReacEq(fe\_values, primary\_variables\_dof[2], R, defMap
      , c\_2, c\_2\_conv, c\_2\_flux, c\_2\_reaction);
*
\end{DoxyCode}
 Please note the above functions will not do the the boundary integration (in this example we have trivial neumman boundary condition). To apply neumman boundary conditions please check {\ttfamily Residual} class which offer a easy way to apply a variety of tpye of neumman boundary conditions. \hypertarget{brain_morph_mechanics}{}\subsubsection{Finite strain mechanics using Neo\-Hookean model}\label{brain_morph_mechanics}
For all mechanics problem we first need to define stress, strain and have a deformation map from which we obtain the strain 
\begin{DoxyCode}
            dealii::Table<3, Sacado::Fad::DFad<double> > P(n\_q\_points,dim,dim);
            dealii::Table<3, Sacado::Fad::DFad<double> > Fe(n\_q\_points,dim,dim);
            deformationMap<Sacado::Fad::DFad<double>, dim> defMap(n\_q\_points); 
            getDeformationMap(fe\_values, primary\_variables\_dof[0], ULocal, defMap);
*
\end{DoxyCode}
 In this example, we consider no growth and isotropic growth\-: 
\begin{DoxyCode}
            \textcolor{keywordflow}{for}(\textcolor{keywordtype}{unsigned} \textcolor{keywordtype}{int} q=0; q<n\_q\_points;q++)\{
                \textcolor{keywordflow}{for} (\textcolor{keywordtype}{unsigned} \textcolor{keywordtype}{int} i=0; i<dim; ++i)\{
                    \textcolor{keywordflow}{for} (\textcolor{keywordtype}{unsigned} \textcolor{keywordtype}{int} j=0; j<dim; ++j)\{
                        Fe[q][i][j]=0.0;
                    \}
                \}
                \textcolor{keywordflow}{if} (c\_1\_0[q] > 0)\{
                    \textcolor{keywordflow}{if}(std::strcmp(GROWTH.c\_str(),\textcolor{stringliteral}{"Uniform"})==0 )\{
                        fac[q] = 1.0; \textcolor{comment}{//Uniform growth}
                    \} 
                    \textcolor{keywordflow}{else} \textcolor{keywordflow}{if}(std::strcmp(GROWTH.c\_str(),\textcolor{stringliteral}{"Isotropic"})==0 )\{
                        fac[q]=std::pow((c\_1[q]/c\_1\_0[q]), 1.0/3.0); \textcolor{comment}{// Isotropic Growth}
                    \} 
                    \textcolor{keywordflow}{else}\{pcout << \textcolor{stringliteral}{"Growth type not supported\(\backslash\)n"}; exit(1);\}
                \}
                \textcolor{keywordflow}{else} \{fac[q] = 1.0;\}
                
            \textcolor{keywordflow}{if} (fac[q] <= 1.0e-15)\{
                    printf(\textcolor{stringliteral}{"*************Non positive growth factor*************. Value %12.4e\(\backslash\)n"}, fac[q].
      val());
            \}
                
            \textcolor{keywordflow}{if} (fac[q] < sat)\{ fac[q] = 1.0; \}
            \textcolor{keywordflow}{else}\{ fac[q] /= sat; \}
                
                \textcolor{keywordflow}{if}(std::strcmp(GROWTH.c\_str(),\textcolor{stringliteral}{"Isotropic"})==0 )\{
                    \textcolor{keywordflow}{for} (\textcolor{keywordtype}{unsigned} \textcolor{keywordtype}{int} i=0; i<dim; ++i)\{
                        \textcolor{keywordflow}{for} (\textcolor{keywordtype}{unsigned} \textcolor{keywordtype}{int} j=0; j<dim; ++j)\{
                        Fe[q][i][j]=defMap.F[q][i][j]/fac[q];
                        \}
                    \}
                \}       
*
\end{DoxyCode}
 After all this we can evaluate stress. For Neohookean model \[ \boldsymbol{P}=\boldsymbol{F}^e\boldsymbol{S}\\ \boldsymbol{S}=0.5\lambda\det(\boldsymbol{C})\boldsymbol{C}^{-1}-(0.5\lambda+\mu)\boldsymbol{C}^{-1}+\mu\boldsymbol{1} \] where $\boldsymbol{C}=\boldsymbol{F}^e(\boldsymbol{F}^e)^{T}$. If young's modulus and Possion ratio are provided, we need to set the Lame parameters first and evaluate the stress 
\begin{DoxyCode}
Residual<vectorType,dim>::setLameParametersByYoungsModulusPoissonRatio(youngsModulus, poissonRatio);
Residual<vectorType,dim>::evaluateNeoHookeanStress(P, Fe);
*
\end{DoxyCode}
 The weak form of elasticity problem is \[ \mathscr{R}_u=\int_{\Omega_{\text{e}}}\nabla \boldsymbol{w}\boldsymbol{T} \text{d}v- \int_{s}\boldsymbol{w} \boldsymbol{f} \cdot \boldsymbol{n} \text{d}s = \boldsymbol{0} \] Again we just simply call the corresponding function to evalue the residual function for $\boldsymbol{u}$. 
\begin{DoxyCode}
Residual<vectorType,dim>::residualForMechanics(fe\_values, primary\_variables\_dof[0], R, P);
*
\end{DoxyCode}
\hypertarget{brain_morph_solve}{}\subsection{Solving nonlinear residual functions}\label{brain_morph_solve}
when we have assembled the system\-\_\-matrix $\frac{\partial \boldsymbol{R}}{\partial \boldsymbol{x}} $, and right hand side vector $-\boldsymbol{R}$, we can solve the Nonlinear residual functions using Newton family method. For dynamics problem we just need to specify the initial and total time, and then call {\ttfamily nonlinear\-Solve} 
\begin{DoxyCode}
  \textcolor{keywordflow}{for} (current\_time=initial\_time; current\_time<=total\_time; current\_time+=dt)\{
    current\_increment++;
    t\_solve = clock();
        solveClass< dim, matrixType, vectorType >::nonlinearSolve(solution);
        solution\_prev=solution;
 \}
*
\end{DoxyCode}
 in the {\ttfamily solve\-Class}, it will take the system\-\_\-matrix and right hand side vector generated by {\ttfamily update\-Linear\-System} we overloaded.\par
 \par
 Now let us discuss the parameter managements and how to choose parameters in solvers. \hypertarget{brain_morph_parameter}{}\subsection{Parameterhandler}\label{brain_morph_parameter}
Recall that \[ \boldsymbol{j}_1^{diff}=-M_1\nabla C_\text{1}; \quad \boldsymbol{j}_2^{diff}=-M_2\nabla C_\text{2}\\ \] \[ r_1= reac_{10} \\ r_1= reac_{20} \] In this example (and almost all real applications), there are many parameters. Deal\-Shell code use {\ttfamily Parameterhandler} to manager all parameters, and {\ttfamily Parameterhandler} is also used in this example to manager the parameters. {\ttfamily Parameterhandler} allows us to modify parameters at runtime without re-\/compiling the code. To use {\ttfamily Parameterhandler} we first need to declare all parameters we will use. Please note parameters used in deal\-Shell are already declared so we only need to declare parameters as we need. In this example we nee 
\begin{DoxyCode}
    params->declare\_entry(\textcolor{stringliteral}{"dt"},\textcolor{stringliteral}{"0"},Patterns::Double() );
    params->declare\_entry(\textcolor{stringliteral}{"totalTime"},\textcolor{stringliteral}{"0"},Patterns::Double() );
    
    params->declare\_entry(\textcolor{stringliteral}{"mesh"},\textcolor{stringliteral}{"1"},Patterns::FileName() );
    params->declare\_entry(\textcolor{stringliteral}{"output\_directory"},\textcolor{stringliteral}{"1"},Patterns::DirectoryName() );
    params->declare\_entry(\textcolor{stringliteral}{"snapshot\_directory"},\textcolor{stringliteral}{"1"},Patterns::DirectoryName() );
    
    \textcolor{comment}{//declare paramters for mechanics}
    params->enter\_subsection(\textcolor{stringliteral}{"Mechanics"});
    params->declare\_entry(\textcolor{stringliteral}{"youngsModulus"},\textcolor{stringliteral}{"0"},Patterns::Double() );
    params->declare\_entry(\textcolor{stringliteral}{"poissonRatio"},\textcolor{stringliteral}{"0"},Patterns::Double() );
    params->declare\_entry(\textcolor{stringliteral}{"saturation\_matID\_0"},\textcolor{stringliteral}{"0"},Patterns::Double() );
    params->declare\_entry(\textcolor{stringliteral}{"saturation\_matID\_1"},\textcolor{stringliteral}{"0"},Patterns::Double() );
    
    params->declare\_entry(\textcolor{stringliteral}{"GROWTH"},\textcolor{stringliteral}{"Isotropic"},Patterns::Selection(\textcolor{stringliteral}{"Uniform|Isotropic"}) );
    params->leave\_subsection(); 
    
    \textcolor{comment}{//declare paramters for concentrations}
    params->enter\_subsection(\textcolor{stringliteral}{"Concentration"});
    params->declare\_entry(\textcolor{stringliteral}{"c1\_ini"},\textcolor{stringliteral}{"0"},Patterns::Double() );
    params->declare\_entry(\textcolor{stringliteral}{"c2\_ini"},\textcolor{stringliteral}{"0"},Patterns::Double() );
    params->declare\_entry(\textcolor{stringliteral}{"c1\_ini\_interface"},\textcolor{stringliteral}{"0"},Patterns::Double() );
    params->declare\_entry(\textcolor{stringliteral}{"c2\_ini\_interface"},\textcolor{stringliteral}{"0"},Patterns::Double() );
    
    params->declare\_entry(\textcolor{stringliteral}{"mobility\_c1"},\textcolor{stringliteral}{"0"},Patterns::Double() );
    params->declare\_entry(\textcolor{stringliteral}{"mobility\_c2"},\textcolor{stringliteral}{"0"},Patterns::Double() );
    params->declare\_entry(\textcolor{stringliteral}{"reac\_10"},\textcolor{stringliteral}{"0"},Patterns::Double() );
    params->declare\_entry(\textcolor{stringliteral}{"reac\_20"},\textcolor{stringliteral}{"0"},Patterns::Double() );
    params->leave\_subsection(); 
*
\end{DoxyCode}
 Then we need to create a {\ttfamily }.prm file to set values for all parameters. Also we need to set parameters for solvers in the file. 
\begin{DoxyCode}
\textcolor{preprocessor}{#parameters file}
\textcolor{preprocessor}{}
\textcolor{preprocessor}{#set global parameters}
\textcolor{preprocessor}{}set dt = 0.1
set totalTime = 1

set mesh = /Users/wzhenlin/GitLab/researchCode/brainMorph/mesh/testMesh.msh
set output\_directory = output/
set snapshot\_directory = snapshot/
\textcolor{preprocessor}{# set equations <name />type(not useful now)}
\textcolor{preprocessor}{}
\textcolor{preprocessor}{#set Mechanics = NeoHookean finite strain}
\textcolor{preprocessor}{}\textcolor{preprocessor}{#set Concentration = Diffusion reaction}
\textcolor{preprocessor}{}

subsection Mechanics
        set youngsModulus =  5.0e3
        set poissonRatio =  0.45
            
        set saturation\_matID\_0 = 10
        set saturation\_matID\_1 = 1
        
     set GROWTH = Tangential
\textcolor{preprocessor}{        #set GROWTH = Isotropic}
\textcolor{preprocessor}{}end

subsection Concentration
        set c1\_ini = 0.5
        set c2\_ini = 0.5
        set c1\_ini\_interface = 1
        set c2\_ini\_interface = 1
                        
        set mobility\_c1 = 0.1
     set mobility\_c2 = 0.1
     set reac\_10 = 0
        set reac\_20 = 0

end
                        
\textcolor{preprocessor}{#}
\textcolor{preprocessor}{}\textcolor{preprocessor}{# parameters reserved for deal.ii first level code:}
\textcolor{preprocessor}{}\textcolor{preprocessor}{#nonLinear\_method : classicNewton}
\textcolor{preprocessor}{}\textcolor{preprocessor}{#solver\_method (direct) : PETScsuperLU, PETScMUMPS}
\textcolor{preprocessor}{}\textcolor{preprocessor}{#solver\_method (iterative) : PETScGMRES PETScBoomerAMG}
\textcolor{preprocessor}{}\textcolor{preprocessor}{#relative\_norm\_tolerance, absolute\_norm\_tolerance, max\_iterations}
\textcolor{preprocessor}{}\textcolor{preprocessor}{#}
\textcolor{preprocessor}{}subsection Nonlinear\_solver
        set nonLinear\_method = classicNewton
        set relative\_norm\_tolerance = 5.0e-12
        set absolute\_norm\_tolerance = 5.0e-12
        set max\_iterations = 50
end
                        
subsection Linear\_solver
        set solver\_method = PETScMUMPS
        set system\_matrix\_symmetricFlag = \textcolor{keyword}{false} # \textcolor{keywordflow}{default} is \textcolor{keyword}{false}
end
*
\end{DoxyCode}
 Alteravely we can use G\-U\-I to set the parameters. \hypertarget{brain_morph_lib}{}\section{mechano\-Chem\-F\-E\-M lib}\label{brain_morph_lib}
To use mechano\-Chem\-F\-E\-M lib, we need to define a class derived from {\ttfamily solve\-Class$<$int dim, class matrix\-Type, class vector\-Type$>$} and {\ttfamily hp\-F\-E\-M$<$int dim$>$}, so that we can used pre-\/defined variables, functions, and overload the abstract virtual function {\ttfamily update\-Linear\-System()} in parent class. The last reason make the inherientance necessary. Also we usually include the following three classes 
\begin{DoxyCode}
ParameterHandler* params;   
Residual<Sacado::Fad::DFad<double>,dim>* ResidualEq;
FEMdata<dim, PETScWrappers::MPI::Vector>* FEMdata\_out;
\end{DoxyCode}
 \hypertarget{growth_results}{}\section{Results}\label{growth_results}
    \hypertarget{brain_morph_com}{}\section{Complete code}\label{brain_morph_com}
The complete implementaion can be found at \href{https://github.com/mechanoChem/mechanoChemFEM/tree/example/brainMorph}{\tt Github}. 