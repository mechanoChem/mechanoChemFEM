\hypertarget{growth_Introduction}{}\section{Introduction}\label{growth_Introduction}
We solve two diffusion reaction equations\-: \[ \frac{\partial C_\text{1}}{\partial t}+\nabla\cdot\boldsymbol{j}_1=r_1 \\ \frac{\partial C_\text{2}}{\partial t}+\nabla\cdot\boldsymbol{j}_2=r_2 \] where $\boldsymbol{j}_1 $ and $\boldsymbol{j}_2 $ are flux terms\-: \[ \boldsymbol{j}_1=-M_1\nabla C_\text{1}; \quad \boldsymbol{j}_2=-M_2\nabla C_\text{2}\\ \] $r_1$ and $r_2$ are reaction terms\-: \[ r_1= R_{10}+R_{11}C_1+R_{13}C_1^2C_2; \quad r_1= R_{20}+R_{21}C_1^2C_2 \] The boundary condiiton is \[ \boldsymbol{j}_1\cdot\boldsymbol{n}=j_n \text{ on }\Gamma_2;\quad \quad \boldsymbol{j}_1\cdot\boldsymbol{n}=0 \text{ on }\Gamma \backslash \Gamma_2; \quad \quad \boldsymbol{j}_2\cdot\boldsymbol{n}=0 \text{ on }\Gamma \] The coupled diffusion-\/reaction equations for two species follow Schnakenberg kinetics. For an activator-\/inhibitor species pair, these equations use auto-\/inhibition with cross-\/activation of a short range species, and auto-\/activation with cross-\/inhibition of a long range species to form so-\/called Turing patterns.\hypertarget{growth_imple}{}\section{Implementation}\label{growth_imple}
We first define the two scalar primary variables\-: 
\begin{DoxyCode}
std::vector<std::vector<std::string> > primary\_variables(2);        
      primary\_variables[0].push\_back(\textcolor{stringliteral}{"c1"}); primary\_variables[0].push\_back(\textcolor{stringliteral}{"component\_is\_scalar"});
      primary\_variables[1].push\_back(\textcolor{stringliteral}{"c2"}); primary\_variables[1].push\_back(\textcolor{stringliteral}{"component\_is\_scalar"});
\end{DoxyCode}
 and we solve both species in one domain. We define the domain and basis order for each primal variables\-: 
\begin{DoxyCode}
\textcolor{keywordtype}{int} number\_domain=1;
\textcolor{keywordtype}{int} diff\_degree=1;
std::vector<std::vector<int> > FE\_support(number\_domain);\textcolor{comment}{// store order of polynomial basis functions, 0
       means FE\_Nothing   }
FE\_support[0].push\_back(diff\_degree);
FE\_support[0].push\_back(diff\_degree);
\end{DoxyCode}
 Before launching the \href{../html/classinit_bound_val_probs.html}{\tt init\-Bound\-Val\-Probs}, we need to initialize the {\bfseries Parameter\-Handler} and declare all paramters we may use\-: 
\begin{DoxyCode}
ParameterHandler params;
params.enter\_subsection(\textcolor{stringliteral}{"Concentration"});   
params.declare\_entry(\textcolor{stringliteral}{"D\_1"},\textcolor{stringliteral}{"0"},Patterns::Double() );
params.declare\_entry(\textcolor{stringliteral}{"D\_2"},\textcolor{stringliteral}{"0"},Patterns::Double() );
\textcolor{comment}{//... more parameters }
params.leave\_subsection();  
\end{DoxyCode}
 Now we just need to have class inherited from \href{../html/classinit_bound_val_probs.html}{\tt init\-Bound\-Val\-Probs} class, and overload the \href{../html/classinit_bound_val_probs.html#ac8f2c3e2a1040c70b709900dc3dfdaea}{\tt get\-\_\-residual()} function\-: 
\begin{DoxyCode}
\textcolor{keyword}{template} <\textcolor{keywordtype}{int} dim>
\textcolor{keyword}{class }diffusion\_reaction: \textcolor{keyword}{public} initBoundValProbs<dim>
\{
    \textcolor{keyword}{public}:
        diffusion\_reaction(std::vector<std::vector<std::string> > \_primary\_variables, std::vector<
      std::vector<int> > \_FE\_support, ParameterHandler& \_params);
        \textcolor{comment}{//this is a overloaded function }
        \textcolor{keywordtype}{void} get_residual(\textcolor{keyword}{const} \textcolor{keyword}{typename} hp::DoFHandler<dim>::active\_cell\_iterator &cell, \textcolor{keyword}{const} 
      FEValues<dim>& fe\_values, Table<1, Sacado::Fad::DFad<double> >& R, Table<1, Sacado::Fad::DFad<double>>& ULocal, 
      Table<1, double >& ULocalConv);
        ParameterHandler* params;       
\};
\end{DoxyCode}
 In the overloaded {\bfseries get\-\_\-residual} function, we define the residual for our problem. As our equations are standrad diffusion-\/reaction, we can simiply use the pre-\/defined model\-: 
\begin{DoxyCode}
this->ResidualEq.residualForDiff_ReacEq(fe\_values, c\_1\_dof, R, c\_1, c\_1\_conv, j\_c\_1, reaction\_1);
this->ResidualEq.residualForDiff_ReacEq(fe\_values, c\_2\_dof, R, c\_2, c\_2\_conv, j\_c\_2, reaction\_2);
\end{DoxyCode}
 Though before we call these two functions, we need the flux and reactions terms, we need to first evaluate the values of the primary fields and their spatial gradients. We also need to evaluate the value of primary fields at previous time step for the Backward Euler time scheme\-: 
\begin{DoxyCode}
dealii::Table<1,double>  c\_1\_conv(n\_q\_points), c\_2\_conv(n\_q\_points);
dealii::Table<1,Sacado::Fad::DFad<double> > c\_1(n\_q\_points), c\_2(n\_q\_points);
dealii::Table<2,Sacado::Fad::DFad<double> >  c\_1\_grad(n\_q\_points, dim), c\_2\_grad(n\_q\_points, dim);

evaluateScalarFunction<double,dim>(fe\_values, c\_1\_dof, ULocalConv, c\_1\_conv);
evaluateScalarFunction<Sacado::Fad::DFad<double>,dim>(fe\_values, c\_1\_dof, ULocal, c\_1); 
evaluateScalarFunctionGradient<Sacado::Fad::DFad<double>,dim>(fe\_values, c\_1\_dof, ULocal, c\_1\_grad);

evaluateScalarFunction<double,dim>(fe\_values, c\_2\_dof, ULocalConv, c\_2\_conv);
evaluateScalarFunction<Sacado::Fad::DFad<double>,dim>(fe\_values, c\_2\_dof, ULocal, c\_2); 
evaluateScalarFunctionGradient<Sacado::Fad::DFad<double>,dim>(fe\_values, c\_2\_dof, ULocal, c\_2\_grad);


\textcolor{comment}{//evaluate diffusion and reaction term}
dealii::Table<1,Sacado::Fad::DFad<double> > reaction\_1(n\_q\_points), reaction\_2(n\_q\_points);
dealii::Table<2,Sacado::Fad::DFad<double> > j\_c\_1(n\_q\_points, dim),j\_c\_2(n\_q\_points, dim);

j\_c\_1=table\_scaling<Sacado::Fad::DFad<double>, dim>(c\_1\_grad,-D\_1);\textcolor{comment}{//-D\_1*c\_1\_grad}
j\_c\_2=table\_scaling<Sacado::Fad::DFad<double>, dim>(c\_2\_grad,-D\_2);\textcolor{comment}{//-D\_2*c\_2\_grad}

\textcolor{keywordflow}{for}(\textcolor{keywordtype}{unsigned} \textcolor{keywordtype}{int} q=0; q<n\_q\_points;q++)\{
    reaction\_1[q]=R\_10+R\_11*c\_1[q]+R\_12*c\_2[q]+R\_13*c\_1[q]*c\_1[q]*c\_2[q];
    reaction\_2[q]=R\_20+R\_21*c\_1[q]+R\_22*c\_2[q]+R\_23*c\_1[q]*c\_1[q]*c\_2[q];
\}
\end{DoxyCode}
 Besides the residual for the P\-D\-Es, we have the boundary conditions on one surface\-: 
\begin{DoxyCode}
\textcolor{keywordflow}{for} (\textcolor{keywordtype}{unsigned} \textcolor{keywordtype}{int} faceID=0; faceID<2*dim; faceID++)\{
    \textcolor{keywordflow}{if}(cell->face(faceID)->boundary\_id()==dim*2 )\{
      FEFaceValues<dim> fe\_face\_values(fe\_values.get\_fe(), *(this->common\_face\_quadrature), update\_values |
       update\_quadrature\_points | update\_JxW\_values);
        fe\_face\_values.reinit(cell,faceID);
        this->ResidualEq.residualForNeummanBC(fe\_values, fe\_face\_values, c\_1\_dof, R, jn);
    \}
\}
\end{DoxyCode}
 The last thing we need to define is the initial condition, we can simpily overload the \href{../html/class_initial_conditions.html#aa10cfdd7350c3810a8deab707f397657}{\tt vector\-\_\-value()} function of the \href{../html/class_initial_conditions.html}{\tt Initial\-Conditions } class, 
\begin{DoxyCode}
\textcolor{keywordtype}{void} InitialConditions<dim>::vector_value (\textcolor{keyword}{const} Point<dim>   &p, Vector<double>   &values)\textcolor{keyword}{ const}\{
  Assert (values.size() == 2, ExcDimensionMismatch (values.size(), 2));
  values(1) = 0;    
  values(0)= 0.5 + 0.1*static\_cast <\textcolor{keywordtype}{double}> (rand())/(static\_cast <double>(RAND\_MAX/2.0))/2;
\}
\end{DoxyCode}
 \hypertarget{growth_results}{}\section{Results}\label{growth_results}
The right plot shows the patterns of the Schnakenberg kinetics.  

The results are generated using paramters shown below. The complete implementaion can be found at \href{https://github.com/mechanoChem/mechanoChemFEM/tree/example/Example1_diffusion_eaction}{\tt Github}.


\begin{DoxyCode}
\textcolor{preprocessor}{#parameters file}
\textcolor{preprocessor}{}
subsection Problem
set print\_parameter = \textcolor{keyword}{true}

set dt = 1
set totalTime = 50
set current\_increment = 0
set off\_output\_index=0
set current\_time = 0
set resuming\_from\_snapshot = \textcolor{keyword}{false}

set output\_directory = output/
set snapshot\_directory = snapshot/

\textcolor{preprocessor}{#FEM}
\textcolor{preprocessor}{}set volume\_quadrature = 3 
set face\_quadrature = 2 
end

subsection Geometry
set X\_0 = 0
set Y\_0 = 0
set Z\_0 = 0
set X\_end = 10
set Y\_end = 10
set Z\_end = 2.0 #no need to 2D

set element\_div\_x=50
set element\_div\_y=50
set element\_div\_z=5 #no need to 2D
end

subsection Concentration

set D\_1 = 0.1
set D\_2 = 4.0
set R\_10 = 0.1
set R\_11 = -1
set R\_13 = 1
set R\_20 = 0.9
set R\_21 = 0
set R\_23 = -1
set jn=-0.01
end
                        
\textcolor{preprocessor}{#}
\textcolor{preprocessor}{}\textcolor{preprocessor}{# parameters reserved for deal.ii first level code:}
\textcolor{preprocessor}{}\textcolor{preprocessor}{#nonLinear\_method : classicNewton}
\textcolor{preprocessor}{}\textcolor{preprocessor}{#solver\_method (direct) : PETScsuperLU, PETScMUMPS}
\textcolor{preprocessor}{}\textcolor{preprocessor}{#solver\_method (iterative) : PETScGMRES PETScBoomerAMG}
\textcolor{preprocessor}{}\textcolor{preprocessor}{#relative\_norm\_tolerance, absolute\_norm\_tolerance, max\_iterations}
\textcolor{preprocessor}{}\textcolor{preprocessor}{#}
\textcolor{preprocessor}{}subsection Nonlinear\_solver
        set nonLinear\_method = classicNewton
        set relative\_norm\_tolerance = 1.0e-12
        set absolute\_norm\_tolerance = 1.0e-12
        set max\_iterations = 20
end
                        
subsection Linear\_solver
        set solver\_method = PETScsuperLU
        set system\_matrix\_symmetricFlag = \textcolor{keyword}{false} # \textcolor{keywordflow}{default} is \textcolor{keyword}{false}
end
\end{DoxyCode}
 