\hypertarget{growth_Introduction}{}\section{Introduction}\label{growth_Introduction}
The Cahn-\/\+Hilliard equation is\+: \[ \frac{\partial C}{\partial t}=\nabla\cdot(M\nabla\mu) \] \[ \mu=\frac{\partial g}{\partial C}-\nabla\cdot k\nabla C \] where $g$ is a non-\/convex, \`{}`homogeneous'\textquotesingle{} free energy density function, whose form has been chosen \[ g(C)=\omega(C-C_\alpha)^2(C-C_\beta)^2 \] The boundary condiiton is \[ \nabla\mu\cdot\boldsymbol{n}=0;\quad \nabla C\cdot\boldsymbol{n}=0 \text{ on }\Gamma \] The double-\/well non-\/convex free energy density function, $g(C)$, drives segregation of the system into two distinct types.

Attention is called to the well-\/known fourth-\/order nature of this partial differential equation in the concentration $C$. The polynomial basis can only achieve C0 continuity across the element. To overcome this difficuity, we split the equation into two equations\+: \[ \frac{\partial C}{\partial t}+\nabla\cdot(-M\nabla\mu)=0 \] \[ k\nabla^2 C=\frac{\partial g}{\partial C}-\mu \] The first equation is diffusion equation, and the second one the Possion equation. \hypertarget{growth_imple}{}\section{Implementation\+: level 1 user}\label{growth_imple}
We first define the two scalar primary variables\+: 
\begin{DoxyCode}{0}
\DoxyCodeLine{std::vector<std::vector<std::string> > primary\_variables(2);        }
\DoxyCodeLine{      primary\_variables[0].push\_back(\textcolor{stringliteral}{"c1"}); primary\_variables[0].push\_back(\textcolor{stringliteral}{"component\_is\_scalar"});}
\DoxyCodeLine{      primary\_variables[1].push\_back(\textcolor{stringliteral}{"mu"}); primary\_variables[1].push\_back(\textcolor{stringliteral}{"component\_is\_scalar"});}
\end{DoxyCode}
 and we solve both species in one domain. We define the domain and basis order for each primal variables\+: 
\begin{DoxyCode}{0}
\DoxyCodeLine{\textcolor{keywordtype}{int} number\_domain=1;}
\DoxyCodeLine{\textcolor{keywordtype}{int} diff\_degree=1;}
\DoxyCodeLine{std::vector<std::vector<int> > FE\_support(number\_domain);\textcolor{comment}{// store order of polynomial basis functions, 0 means FE\_Nothing   }}
\DoxyCodeLine{FE\_support[0].push\_back(diff\_degree);}
\DoxyCodeLine{FE\_support[0].push\_back(diff\_degree);}
\end{DoxyCode}
 Before launching the \href{../html/classinit_bound_val_probs.html}{\texttt{ init\+Bound\+Val\+Probs}}, we need to initialize the {\bfseries{Parameter\+Handler}} and declare all paramters we may use\+: 
\begin{DoxyCode}{0}
\DoxyCodeLine{ParameterHandler params;}
\DoxyCodeLine{params.enter\_subsection(\textcolor{stringliteral}{"Concentration"});   }
\DoxyCodeLine{params.declare\_entry(\textcolor{stringliteral}{"omega"},\textcolor{stringliteral}{"0"},Patterns::Double() );}
\DoxyCodeLine{params.declare\_entry(\textcolor{stringliteral}{"c\_alpha"},\textcolor{stringliteral}{"0"},Patterns::Double() );}
\DoxyCodeLine{\textcolor{comment}{//... more parameters }}
\DoxyCodeLine{params.leave\_subsection();  }
\end{DoxyCode}
 Now we just need to have class inherited from \href{../html/classinit_bound_val_probs.html}{\texttt{ init\+Bound\+Val\+Probs}} class, and overload the \href{../html/classinit_bound_val_probs.html\#ac8f2c3e2a1040c70b709900dc3dfdaea}{\texttt{ get\+\_\+residual()}} function\+: 
\begin{DoxyCode}{0}
\DoxyCodeLine{\textcolor{keyword}{template} <\textcolor{keywordtype}{int} dim>}
\DoxyCodeLine{\textcolor{keyword}{class }CahnHilliard: \textcolor{keyword}{public} initBoundValProbs<dim>}
\DoxyCodeLine{\{}
\DoxyCodeLine{    \textcolor{keyword}{public}:}
\DoxyCodeLine{        CahnHilliard(std::vector<std::vector<std::string> > \_primary\_variables, std::vector<std::vector<int> > \_FE\_support, ParameterHandler\& \_params);}
\DoxyCodeLine{        \textcolor{comment}{//this is a overloaded function }}
\DoxyCodeLine{        \textcolor{keywordtype}{void} get\_residual(\textcolor{keyword}{const} \textcolor{keyword}{typename} hp::DoFHandler<dim>::active\_cell\_iterator \&cell, \textcolor{keyword}{const} FEValues<dim>\& fe\_values, Table<1, Sacado::Fad::DFad<double> >\& R, Table<1, Sacado::Fad::DFad<double>>\& ULocal, Table<1, double >\& ULocalConv);}
\DoxyCodeLine{        ParameterHandler* params;       }
\DoxyCodeLine{\};}
\end{DoxyCode}
 In the overloaded {\bfseries{get\+\_\+residual}} function, we define the residual for our problem. As our equations are one standrad diffusion equation and Possion equation, we can simiply use the pre-\/defined model\+: 
\begin{DoxyCode}{0}
\DoxyCodeLine{this->ResidualEq.residualForDiffusionEq(fe\_values, c\_dof, R, c\_1, c\_1\_conv, j\_c\_1);}
\DoxyCodeLine{this->ResidualEq.residualForPoissonEq(fe\_values, mu\_dof, R, kappa\_c\_1\_grad, rhs\_mu);}
\end{DoxyCode}
 fe\+\_\+values\+: deal.\+ii Finite element evaluated in quadrature points of a cell.

c\+\_\+1\+\_\+dof, mu\+\_\+dof\+: dof index for two primial variables.

R\+: residual vector of a cell.

c\+\_\+1\+: concentration at current time step.

c\+\_\+1\+\_\+conv\+: concentration at the last time step.

j\+\_\+c\+\_\+1\+: flux of the species.

kappa\+\_\+c\+\_\+1\+\_\+grad\+:left hand side of the Possion equation.

rhs\+\_\+mu\+: right hand side of the Possion equation.



 The following code deomonstrate how to evaluate the variables defined above. 
\begin{DoxyCode}{0}
\DoxyCodeLine{dealii::Table<1,double>  c\_1\_conv(n\_q\_points);}
\DoxyCodeLine{dealii::Table<1,Sacado::Fad::DFad<double> > c\_1(n\_q\_points), mu(n\_q\_points);}
\DoxyCodeLine{dealii::Table<2,Sacado::Fad::DFad<double> >  c\_1\_grad(n\_q\_points, dim), mu\_grad(n\_q\_points, dim);}
\DoxyCodeLine{}
\DoxyCodeLine{evaluateScalarFunction<double,dim>(fe\_values, c\_dof, ULocalConv, c\_1\_conv);}
\DoxyCodeLine{evaluateScalarFunction<Sacado::Fad::DFad<double>,dim>(fe\_values, c\_dof, ULocal, c\_1);   }
\DoxyCodeLine{evaluateScalarFunctionGradient<Sacado::Fad::DFad<double>,dim>(fe\_values, c\_dof, ULocal, c\_1\_grad);}
\DoxyCodeLine{}
\DoxyCodeLine{evaluateScalarFunction<Sacado::Fad::DFad<double>,dim>(fe\_values, mu\_dof, ULocal, mu);   }
\DoxyCodeLine{evaluateScalarFunctionGradient<Sacado::Fad::DFad<double>,dim>(fe\_values, mu\_dof, ULocal, mu\_grad);}
\DoxyCodeLine{}
\DoxyCodeLine{\textcolor{comment}{//evaluate diffusion and reaction term}}
\DoxyCodeLine{dealii::Table<1,Sacado::Fad::DFad<double> > rhs\_mu(n\_q\_points);}
\DoxyCodeLine{dealii::Table<2,Sacado::Fad::DFad<double> > j\_c\_1(n\_q\_points, dim), kappa\_c\_1\_grad(n\_q\_points, dim);}
\DoxyCodeLine{}
\DoxyCodeLine{j\_c\_1=table\_scaling<Sacado::Fad::DFad<double>, dim>(mu\_grad,-M);\textcolor{comment}{//-D\_1*c\_1\_grad}}
\DoxyCodeLine{kappa\_c\_1\_grad=table\_scaling<Sacado::Fad::DFad<double>, dim>(c\_1\_grad,kappa);}
\DoxyCodeLine{}
\DoxyCodeLine{\textcolor{keywordflow}{for}(\textcolor{keywordtype}{unsigned} \textcolor{keywordtype}{int} q=0; q<n\_q\_points;q++) rhs\_mu[q]=2*omega*(c\_1[q]-c\_alpha)*(c\_1[q]-c\_beta)*(2*c\_1[q]-c\_alpha-c\_beta)-mu[q];}
\end{DoxyCode}


The last thing we need to define is the initial condition, we can simpily overload the \href{../html/class_initial_conditions.html\#aa10cfdd7350c3810a8deab707f397657}{\texttt{ vector\+\_\+value()}} function of the \href{../html/class_initial_conditions.html}{\texttt{ Initial\+Conditions }} class, 
\begin{DoxyCode}{0}
\DoxyCodeLine{\textcolor{keywordtype}{void} InitialConditions<dim>::vector\_value (\textcolor{keyword}{const} Point<dim>   \&p, Vector<double>   \&values)\textcolor{keyword}{ const}\{}
\DoxyCodeLine{  Assert (values.size() == 2, ExcDimensionMismatch (values.size(), 2));}
\DoxyCodeLine{  values(1) = 0;    }
\DoxyCodeLine{ values(0)= 0.5 + 0.04*(static\_cast <double> (rand())/(static\_cast <double>(RAND\_MAX))-0.5);}
\DoxyCodeLine{\}}
\DoxyCodeLine{}
\DoxyCodeLine{  The complete implementaion can be found at  <a href=\textcolor{stringliteral}{"https://github.com/mechanoChem/mechanoChemFEM/tree/example/Example2\_CahnHilliard"}>Github</a>. }
\end{DoxyCode}
\hypertarget{growth_file}{}\section{Parameter file\+: interface for level 2 user}\label{growth_file}

\begin{DoxyCode}{0}
\DoxyCodeLine{\textcolor{preprocessor}{\#parameters file}}
\DoxyCodeLine{}
\DoxyCodeLine{subsection Problem}
\DoxyCodeLine{set print\_parameter = \textcolor{keyword}{true}}
\DoxyCodeLine{}
\DoxyCodeLine{set dt = 5}
\DoxyCodeLine{set totalTime = 250}
\DoxyCodeLine{set current\_increment = 0}
\DoxyCodeLine{set off\_output\_index=0}
\DoxyCodeLine{set current\_time = 0}
\DoxyCodeLine{set resuming\_from\_snapshot = \textcolor{keyword}{false}}
\DoxyCodeLine{}
\DoxyCodeLine{set output\_directory = output/}
\DoxyCodeLine{set snapshot\_directory = snapshot/}
\DoxyCodeLine{}
\DoxyCodeLine{\textcolor{preprocessor}{\#FEM}}
\DoxyCodeLine{set volume\_quadrature = 3 }
\DoxyCodeLine{set face\_quadrature = 2 }
\DoxyCodeLine{}
\DoxyCodeLine{end}
\DoxyCodeLine{}
\DoxyCodeLine{subsection Geometry}
\DoxyCodeLine{set X\_0 = 0}
\DoxyCodeLine{set Y\_0 = 0}
\DoxyCodeLine{set Z\_0 = 0}
\DoxyCodeLine{set X\_end = 1 }
\DoxyCodeLine{set Y\_end = 1}
\DoxyCodeLine{set Z\_end = 2.0 \#no need to 2D}
\DoxyCodeLine{}
\DoxyCodeLine{set element\_div\_x=50}
\DoxyCodeLine{set element\_div\_y=50}
\DoxyCodeLine{set element\_div\_z=5 \#no need to 2D}
\DoxyCodeLine{end}
\DoxyCodeLine{}
\DoxyCodeLine{subsection Concentration}
\DoxyCodeLine{set omega = 0.25}
\DoxyCodeLine{set c\_alpha = 0.2}
\DoxyCodeLine{set c\_beta = 0.8}
\DoxyCodeLine{set kappa = 0.002}
\DoxyCodeLine{set M=0.1}
\DoxyCodeLine{end}
\DoxyCodeLine{                        }
\DoxyCodeLine{\textcolor{preprocessor}{\#}}
\DoxyCodeLine{\textcolor{preprocessor}{\# parameters reserved for deal.ii first level code:}}
\DoxyCodeLine{\textcolor{preprocessor}{\#nonLinear\_method : classicNewton}}
\DoxyCodeLine{\textcolor{preprocessor}{\#solver\_method (direct) : PETScsuperLU, PETScMUMPS}}
\DoxyCodeLine{\textcolor{preprocessor}{\#solver\_method (iterative) : PETScGMRES PETScBoomerAMG}}
\DoxyCodeLine{\textcolor{preprocessor}{\#relative\_norm\_tolerance, absolute\_norm\_tolerance, max\_iterations}}
\DoxyCodeLine{\textcolor{preprocessor}{\#}}
\DoxyCodeLine{subsection Nonlinear\_solver}
\DoxyCodeLine{        set nonLinear\_method = classicNewton}
\DoxyCodeLine{        set relative\_norm\_tolerance = 1.0e-12}
\DoxyCodeLine{        set absolute\_norm\_tolerance = 1.0e-12}
\DoxyCodeLine{        set max\_iterations = 10}
\DoxyCodeLine{end}
\DoxyCodeLine{                        }
\DoxyCodeLine{subsection Linear\_solver}
\DoxyCodeLine{        set solver\_method = PETScsuperLU}
\DoxyCodeLine{        set system\_matrix\_symmetricFlag = \textcolor{keyword}{false} \# \textcolor{keywordflow}{default} is \textcolor{keyword}{false}}
\DoxyCodeLine{end}
\end{DoxyCode}
\hypertarget{growth_results}{}\section{Results}\label{growth_results}
The results are generated using paramters shown above.

  