\hypertarget{growth_Introduction}{}\section{Introduction}\label{growth_Introduction}
A battery cell consists of porous, positive and negative electrodes, a separator and current collectors(see Figure 1). The electrode consists of active-\/material particles and carbon-\/binders, and the porespace is filled with the electrolyte. The two electrodes are isolated by a separator which is usuallymade of polyolefin for Li-\/ion batteries. The porous separator is perfused with electrolyte, allowingionic transport between the electrodes. Metallic current collectors are located at either end of the battery.

This example present a coupled continuum formulation for the electrostatic, chemical, thermal, mechanical and fluid physics in battery materials. The treatment is at the particle scale, at whic hthe active particles held together by binders, the porous separator, current collectors and the perfusing electrolyte are explicitly modeled.

This code is used to generate results for paper\-: {\bfseries A multi-\/physics battery model with particle scale resolution ofporosity evolution driven by intercalation strain and electrolyte flow}, Z. Wang, K. Garikipati, Journal of the Electrochemical Society, Vol. 165\-: A2421-\/\-A438, 2018, doi\-:10.\-1149/2.0141811jes  \hypertarget{battery_particle_section1}{}\subsection{The multi-\/physics particle scale model}\label{battery_particle_section1}
We lay down the governing equations for primary variables in three dimensions. As is the convention in continuum mechanics, the solids (current collector,active-\/material particle and polymeric separator particles) are described in a Lagrangian setting, while the fluid (electrolyte) flow is described in an Eulerian setting. Electrolytic flow around thedeforming solid components alters the fluid domain. The fluid mesh therefore must be remapped asthe computation proceeds. The Arbitrary Lagrangian-\/\-Eulerian (A\-L\-E) framework is adopted herefor this purpose. \hypertarget{battery_particle_subsub1}{}\subsubsection{Electro-\/chemo-\/thermal equations}\label{battery_particle_subsub1}
In the active material we have conservation of lithium\-: \[ \frac{\partial C_\text{Li}}{\partial t}+\nabla\cdot\boldsymbol{j}=0\\ \boldsymbol{j}=-D\nabla C_\text{Li} \] where $D$ is diffusivity. Conservation of of Li $^+$ cations in the electrolyte is written similarly\-: \[ \frac{\partial C_{\text{Li}^+}}{\partial t}+\nabla\cdot\boldsymbol{j}_+=0 \] In general, the flux of each species can depend on the concentration gradients of the other species. This is a special case of the Onsager reciprocity relations with the dependence on chemical potential gradients reduced to concentration gradient dependence. Following Newman and Thomas-\/\-Alyea $^{\text{J. Newman and K. E. Thomas-Alyea,John Wiley and Sons, Inc, (2004).}}$, we have \[ \boldsymbol{j}_+=-D_{+}\nabla C_{\text{Li}^+} +\frac{t_+}{F}\boldsymbol{i}_\text{E}+C_{\text{Li}^+}\boldsymbol{v} \] for a binary solution. Here, $D_{+}$ is the diffusivity, $t_+$ is the transference number of the cation, $\boldsymbol{v}$ is the electrolyte velocity, $F$ is the Faraday constant, and $\boldsymbol{i}_\text{E}$ is the total current in the electrolyte phase which will be derived in what follows.

The total current in the current collector and active material is governed by Ohm's law\-: \[ \boldsymbol{i}_\text{S}=-\sigma_\text{S} \nabla\phi_\text{S} \] where $\phi_\text{S}$ is the electric potential, and the conductivity $\sigma_\text{S}$ differs between the current collector and active material. In the electrolyte we have \[ \boldsymbol{i}_\text{E}=-\sigma_\text{E}\nabla\phi_\text{E}-\gamma_\text{D}\nabla\ln C_{\text{Li}^+} \] where $\sigma_\text{E}$ is the electrolyte's conductivity, and $\gamma_\text{D}$ is the diffusion conductivity evaluated as\-: \[ \gamma_\text{D}=\frac{2R\theta \sigma_\text{E}}{F}\left(1-t_+\right)\left(1+\frac{d \ln f}{d\ln C_{\text{Li}^+}}\right), \] with $f$ being the mean molar activity coefficient of the electrolyte, which is assumed to be constant. Thus the relation simplifies to \[ \gamma_\text{D}=\frac{2 R\theta \sigma_\text{E}}{F}\left(1-t_+\right). \] Under the electroneutrality approximation, we have \[ \nabla\cdot (- \sigma_\text{s}\nabla\phi_\text{S})=0 \] in the active material, and \[ \nabla\cdot\left(-\sigma_\text{E}\nabla\phi_\text{E}-\gamma_\text{D}\nabla\ln C_{\text{Li}^+}\right)=0 \] in the electrolyte. These two equations properly describe the electric potentials with the electroneutrality approximation, such that the double layer effect is neglected.\hypertarget{battery_particle_section2}{}\subsection{The standard thermal equations}\label{battery_particle_section2}
Heat generation and transport are governed by the heat equation, which is derived from the first law of thermodynamics. Since the velocity of the electrolyte is low, with a Peclet number Pe $\sim 1.0\times 10^{-7}$, we neglect heat flux by advection. For the temperature $\theta$, we have the standard form of the heat equation\-: \[ \rho C_p\frac{\text{d}\theta}{\text{d}t}+\nabla \cdot\boldsymbol{q}=0 \] where $\rho$ is the mass density of the electrode, $C_p$ is the specific heat and $\boldsymbol{q}$ is the heat flux. The heat flux can be expressed as \[ \boldsymbol{q}=\phi F\sum_i z_i\boldsymbol{j}_i-\lambda\nabla\theta+\boldsymbol{q}^\text{D} \] where $\lambda$ is the thermal conductivity. The first term on right is associated with Joule heating and the last term $\boldsymbol{q}^\text{D}$ is the Dufour effect, which is ignored in this work. Again, with the electroneutrality approximation we have \$ z\-\_\-i \{j\}\-\_\-i=0\$, we have \[ \rho C_p\frac{\text{d}\theta}{\text{d}t}-\lambda \nabla^2\theta+\nabla\phi_\text{S}\cdot\boldsymbol{i}_\text{S}=0 \] in the electrode, and \[ \rho C_p\frac{\text{d}\theta}{\text{d}t}-\lambda \nabla^2\theta+\nabla\phi_\text{E}\cdot\boldsymbol{i}_\text{E}=0 \] in the electrolyte.\hypertarget{battery_particle_section3}{}\subsection{Finite strain mechanics}\label{battery_particle_section3}
Lithium intercalation and de-\/intercalation induce expansion and contraction, respectively, of the active material. Additionally, the active material, binder, porous separator and current collector undergo thermal expansion. The kinematics of finite strain leads to the following decomposition in the active material\-: \[ \boldsymbol{F}=\boldsymbol{F}^\text{e}\boldsymbol{F}^\text{c}\boldsymbol{F}^{\theta}. \] For other solid sub-\/domains (binder and polymeric separator particles) we have \[ \boldsymbol{F}=\boldsymbol{F}^\text{e}\boldsymbol{F}^{\theta}. \] Here, $\boldsymbol{F} = \boldsymbol{I} + \partial\boldsymbol{u}/\partial\boldsymbol{X}$, is the total deformation gradient tensor in each of the relevant solid regions. Its multiplicative components $\boldsymbol{F}^\text{e}$, $\boldsymbol{F}^\text{c}$ and $\boldsymbol{F}^{\theta}$, are, respectively, the elastic, chemical (induced by lithium intercalation) and thermal components. In the absence of a body force the strong form of the mechanics problem in the current configuration is \[ \nabla\cdot\boldsymbol{T} = \boldsymbol{0}, \] \[ \text{for}\quad \boldsymbol{T}= \frac{1}{\det{\boldsymbol{F}^\text{e}}}\frac{\partial W}{\partial \boldsymbol{F}^\text{e}}\boldsymbol{F}^{\text{e}^\text{T}}, \] where $\boldsymbol{T}$ is the Cauchy stress tensor and $W$ is the strain energy density function. The chemical and thermal expansion components of the multiplicative decomposition of $\boldsymbol{F}$ are modelled as isotropic, e.\-g.\-: \[ F^{\text{c}}_{iJ}=(1+\beta^{\text{c}})^{1/3}\delta_{iJ} \\ F^{\theta}_{iJ}=(1+\beta^{\theta})^{1/3}\delta_{iJ} \] The lithiation swelling response function is parameterized by the lithium concentration. We write \[ \beta^{\theta}(\theta)=\Omega_{\theta} (\theta-\theta_0) \] for the thermal expansion functions.\hypertarget{battery_particle_section4}{}\subsection{Incompressible fluid model}\label{battery_particle_section4}
We model the electrolyte as an incompressible, creeping flow. In this regime, with inertia and body forces being neglected, the Stokes equations are\-: \[ -2\eta\nabla\cdot\varepsilon(\boldsymbol{v})+\nabla p=0 \] \[ \nabla\cdot\boldsymbol{v}=0 \] where $\varepsilon(\boldsymbol{v})=\frac{1}{2}(\nabla\boldsymbol{v}+\nabla\boldsymbol{v}^\text{T})$ is the strain rate tensor, and $\eta$ is the dynamic viscosity.\hypertarget{battery_particle_section5}{}\subsection{The Arbitrary Lagrangian-\/\-Eulerian framework}\label{battery_particle_section5}
In the fluid (electrolyte) phase, whose description is Eulerian, the mesh must be mapped to evolve with the fluid sub-\/domain as it flows around the deforming solid components. If the mesh displacement is $\boldsymbol{u}_\text{m}$, the spatial gradient operator acting on a variable $\tau$ in the Eulerian setting is \[ \nabla \tau=\frac{\partial\tau}{\partial\boldsymbol{X}}\boldsymbol{F}_\text{m}^{-1} \] where $\boldsymbol{F}_\text{m}= \boldsymbol{I} +\partial \boldsymbol{u}_\text{m}/\partial \boldsymbol{X}$, is the deformation gradient tensor of the mesh. In the solid phase, $\boldsymbol{u}_\text{m}$ coincides with the displacement of material points, $\boldsymbol{u}$, and $\boldsymbol{F}_\text{m}$ is replaced by the deformation gradient tensor of the solid, $\boldsymbol{F}$. In the fluid phase the description of fluid mesh deformation could be arbitrary, but its choice proves to be critical in solid-\/fluid interaction problems. The mesh displacement in the fluid phase can be solved by Poisson equations, arbitrary elasticity equations or bi-\/harmonic equations.

The large lithium intercalation strain may lead to a dramatic volume change of the active material. The large deformation of active material may cause extreme distortions of the mesh in the fluid. For this reason, we apply adaptive mesh rezoning schemes. The idea is to introduce a constraint condition over two adjacent element nodes $i$ and $j$ such that the relative difference of net displacements is less than the element length $h^e$\-: \[ |\boldsymbol{u}_m^i-\boldsymbol{u}_m^j|\le\alpha h^e, \] which is equivalent to \[ |\nabla\boldsymbol{u}_m|\le\alpha \] where $\alpha\in[0,1)$ is the tolerance for element distortion. This constraint is applied element wise as a penalty term\-: \[ \nabla^2(1+\tau_m\boldsymbol{u}_m)=\boldsymbol{0} \] where $\tau_m$ is the weight function that imposes spatially varying stiffening effects for the mesh. The key idea for choosing $\tau_m$ is to enforce this value to be high for smaller elements and small for large elements. In this work we choose \[ \tau_m=\left(\frac{\det \boldsymbol{F}_m^0}{\det \boldsymbol{F}_m} \right)^\delta \] where $\boldsymbol{F}_m^0$ is the initial deformation gradient tensor of the mesh and $\delta$ is a constant value.\hypertarget{battery_particle_subsection3}{}\subsection{Boundary and interface condition}\label{battery_particle_subsection3}
The multi-\/physics character of this problem extends to the restriction of certain physics--and therefore the corresponding partial differential equations--to specific sub-\/domains. The conventional application of boundary conditions translates to interface conditions where the sub-\/domains meet. During discharging or charging, the following chemical reactions occur at the interface between the active material and electrolyte\-: \[ \text{Li}\to \text{Li}^++q^-, \quad \text{-ve electrode during discharge/+ve electrode during charge,}\nonumber\\ \text{Li}^++q^- \to\text{Li},\quad \text{+ve electrode during discharge/-ve electrode during charge}.\nonumber \] The reaction rate is given by the Butler-\/\-Volmer model \[ j_n=j_0\left(\text{exp}\left(\frac{\alpha_aF}{R\theta}(\phi_\text{S}-\phi_\text{E}-U)\right)-\text{exp}\left(-\frac{\alpha_aF}{R\theta}(\phi_\text{S}-\phi_\text{E}-U)\right)\right) \label{eq:BVeuqation}\\ j_0=k_0(C_{\text{Li}^+})^{\alpha_a} \frac{(C_\text{Li}^{\text{max}}-C_\text{Li})^{\alpha_a}}{(C_1^{\text{max}})^{\alpha_a}} \frac{(C_\text{Li})^{\alpha_c}}{{(C_1^\text{max}})^{\alpha_c}} \] where $\alpha_a, \alpha_c$ are transfer coefficients and $k_0$ is a kinetic rate constant. The interface condition for lithium and lithium ions\-: \[ \boldsymbol{j}\cdot \boldsymbol{n}_\text{s}=j_n \quad \text{on } \Gamma_\text{s-e},\\ \boldsymbol{j}_+\cdot \boldsymbol{n}_\text{e}=-j_n\ \quad \text{on } \Gamma_\text{s-e}, \] The interface condition for the electric potential\-: \[ \boldsymbol{i}_\text{S}\cdot\boldsymbol{n}_\text{s}=Fj_n \quad \text{on } \Gamma_\text{s-e}\\ \boldsymbol{i}_\text{E}\cdot\boldsymbol{n}_\text{e}=-Fj_n \ \quad \text{on } \Gamma_\text{s-e} \] At the interface, the chemical reactions also generate heat\-: \[ Q_{\Gamma_\text{s-e}}=q_{\text{rxn}}+q_{\text{rev}} \] where \[ q_{\text{rxn}}=Fj_n(\phi_\text{S}-\phi_\text{E}-U) ;\quad \text{irreversible entropic heat,} \\ q_{\text{rev}}=Fj_n\theta\frac{\partial U}{\partial \theta};\quad\text{reversible entropic heat.} \] The electrolytic fluid is bounded within the solid materials\-: active material, binder, polymeric separator particles, and current collector. No-\/slip conditions give \[ \boldsymbol{v}=\dot{\boldsymbol{u}} \quad \text{on } \Gamma_\text{s-e}, \Gamma_\text{e-c}, \Gamma_\text{e-p} \] Traction continuity on all solid-\/fluid interfaces gives\-: \[ \boldsymbol{T}\boldsymbol{n}_\text{s}=-(2\eta\cdot\varepsilon(\boldsymbol{v})-p\boldsymbol{1})\boldsymbol{n}_\text{e} \quad \text{on } \Gamma_\text{s-e} \\ \boldsymbol{T}\boldsymbol{n}_\text{c}=-(2\eta\cdot\varepsilon(\boldsymbol{v})-p\boldsymbol{1})\boldsymbol{n}_\text{e} \quad \text{on } \Gamma_\text{e-c} \\ \boldsymbol{T}\boldsymbol{n}_\text{p}=-(2\eta\cdot\varepsilon(\boldsymbol{v})-p\boldsymbol{1})\boldsymbol{n}_\text{e} \quad \text{on } \Gamma_\text{e-p} \] Compatibility at the solid-\/fliud interfaces equates the A\-L\-E mesh deformation, $\boldsymbol{u}_\text{m}$ in the electrolyte sub-\/domain to the solid deformations\-: \[ \boldsymbol{u}_\text{m}=\boldsymbol{u} \quad \text{on } \Gamma_\text{s-e}, \Gamma_\text{c-e}, \Gamma_\text{p-e}. \]  \hypertarget{battery_particle_Implementation}{}\section{Implementation\-: level 0 developer}\label{battery_particle_Implementation}
The code is very similar to \hyperlink{battery_electrode_scale}{Battery model at electrode scale}. Here we mainly discuss how to apply interface condition and apply constraints during solve step.

{\bfseries Applying boundary and interface conditions}\par
 During updating linear system, we apply these boundary and interface conditions discussed above. For interface coditions, we first need to find these interface. For example to find interface bewteen active\-\_\-material and electrolyte we can loop the element face and check the domain id of elements on both sides. 
\begin{DoxyCode}
\textcolor{keywordflow}{if}(cell->material\_id()==active\_material\_id  )\{
    \textcolor{keywordflow}{for} (\textcolor{keywordtype}{unsigned} \textcolor{keywordtype}{int} f=0; f<GeometryInfo<dim>::faces\_per\_cell; ++f)\{
        \textcolor{keywordflow}{if} (cell->at\_boundary(f) == \textcolor{keyword}{false})\{
            \textcolor{keywordflow}{if}(cell->neighbor(f)->material\_id()==electrolyte\_id  and cell->neighbor(f)->has\_children() == \textcolor{keyword}{
      false})\{
              assemble\_interface\_activeMaterial\_electrolyte\_term(cell, f, ULocal, ULocalConv, localized\_U, 
      R);                                          
            \}
            \textcolor{keywordflow}{else} \textcolor{keywordflow}{if} (cell->neighbor(f)->has\_children() == \textcolor{keyword}{true})\{
                \textcolor{keywordflow}{for} (\textcolor{keywordtype}{unsigned} \textcolor{keywordtype}{int} sf=0; sf<cell->face(f)->n\_children(); ++sf)\{
                    \textcolor{keywordflow}{if} (cell->neighbor\_child\_on\_subface(f, sf)->material\_id()==electrolyte\_id )\{
                      assemble\_interface\_activeMaterial\_electrolyte\_term(cell, f, ULocal, ULocalConv, 
      localized\_U, R);                                                                
                        \textcolor{keywordflow}{break};
                    \}
                \}
            \}
        \}
    \}
\}
\end{DoxyCode}
 Once we find the interface, the corresponding function for assemble interfacial residual is called. Now let's take a look of one of these functions. The following code iitialize the finite elemnt system for the elements on both sides and their faces. 
\begin{DoxyCode}
  hp::FEValues<dim> hp\_fe\_values (fe\_collection, q\_collection, update\_values | update\_quadrature\_points  | 
      update\_JxW\_values | update\_gradients);
FEFaceValues<dim> activeMaterial\_fe\_face\_values (*fe\_system[active\_material\_fe], *common\_face\_quadrature, 
      update\_values | update\_quadrature\_points | update\_JxW\_values | update\_normal\_vectors | update\_gradients);
FEFaceValues<dim> electrolyte\_fe\_face\_values (*fe\_system[electrolyte\_fe], *common\_face\_quadrature, 
      update\_values | update\_quadrature\_points | update\_JxW\_values | update\_normal\_vectors | update\_gradients); 
  
  hp\_fe\_values.reinit (cell);
  \textcolor{keyword}{const} FEValues<dim> &activeMaterial\_fe\_values = hp\_fe\_values.get\_present\_fe\_values();
  hp\_fe\_values.reinit (cell->neighbor(f));
  \textcolor{keyword}{const} FEValues<dim> &electrolyte\_fe\_values = hp\_fe\_values.get\_present\_fe\_values();
  activeMaterial\_fe\_face\_values.reinit(cell, f);
  electrolyte\_fe\_face\_values.reinit(cell->neighbor(f), cell->neighbor\_of\_neighbor(f));
  std::vector<types::global\_dof\_index> activeMaterial\_neighbor\_dof\_indices (electrolyte\_dofs\_per\_cell);
\end{DoxyCode}
 Then we can evaluate all fields at surface quadrature points using \hyperlink{group___evaluation_functions_ga2e2fbeb2173113c6889c73bbb7304789}{evaluate functions } 
\begin{DoxyCode}
evaluateVectorFunctionGradient<Sacado::Fad::DFad<double>,dim>(electrolyte\_fe\_values, 
      electrolyte\_fe\_face\_values, v\_dof, ULocal\_electrolyte, velocity\_grad,defMap\_mesh\_surface);
evaluateScalarFunction<Sacado::Fad::DFad<double>,dim>(electrolyte\_fe\_values, electrolyte\_fe\_face\_values, 
      p\_dof, ULocal\_electrolyte, Pressure);
evaluateScalarFunction<Sacado::Fad::DFad<double>,dim>(electrolyte\_fe\_values, electrolyte\_fe\_face\_values, 
      c\_li\_plus\_dof, ULocal\_electrolyte, c\_li\_plus);
evaluateScalarFunction<Sacado::Fad::DFad<double>,dim>(electrolyte\_fe\_values, electrolyte\_fe\_face\_values, 
      phi\_e\_dof, ULocal\_electrolyte, phi\_e);

evaluateScalarFunction<Sacado::Fad::DFad<double>,dim>(activeMaterial\_fe\_values, 
      activeMaterial\_fe\_face\_values, T\_dof, ULocal, T);
evaluateScalarFunction<Sacado::Fad::DFad<double>,dim>(activeMaterial\_fe\_values, 
      activeMaterial\_fe\_face\_values, c\_li\_dof, ULocal, c\_li);
evaluateScalarFunction<Sacado::Fad::DFad<double>,dim>(activeMaterial\_fe\_values, 
      activeMaterial\_fe\_face\_values, phi\_s\_dof, ULocal, phi\_s);
\end{DoxyCode}
 These interface conditions are Neumman conditions. And the residual is assembled by calling the residual functions as we discuss before.

{\bfseries Applying constraints during solve step}\par
 The interface condition $\boldsymbol{u}_\text{m}=\boldsymbol{u}$ is a master-\/slave interface, meanning $\boldsymbol{u}_\text{m}$ does not affect $\boldsymbol{u}$ but just take its value as a Dirichlet boundary condition. So we cannot use the constraint materix to apply this constraint. Here we apply this constrain manually during the solve step. First we need to find the pair of D\-O\-Fs for $\boldsymbol{u}$ and $\boldsymbol{u}_\text{m}$ at same node. And save it as constrain\-\_\-index. 
\begin{DoxyCode}
\textcolor{keywordflow}{if} (ck1>=0 && ck1<dim )\{
        \textcolor{keywordtype}{int} dof\_per\_node=local\_face\_dof\_indices2.size()/4;
        hpFEM<dim>::constraints.add\_line (local\_face\_dof\_indices1[i]);\textcolor{comment}{//u\_mesh                             
                               }
        \textcolor{keywordflow}{for} (\textcolor{keywordtype}{unsigned} \textcolor{keywordtype}{int} j=(node\_u\_m\_applied/2)*dof\_per\_node; j<local\_face\_dof\_indices2.size(); ++j)\{
            \textcolor{keyword}{const} \textcolor{keywordtype}{unsigned} \textcolor{keywordtype}{int} ck3 = fe\_system[current\_collector\_fe]->face\_system\_to\_component\_index(j).
      first;
            \textcolor{keywordflow}{if}(ck3==ck1)\{
                constrain\_index[0].push\_back(local\_face\_dof\_indices1[i]);
                constrain\_index[1].push\_back(local\_face\_dof\_indices2[j]);
                node\_u\_m\_applied++;
                \textcolor{keywordflow}{break};
            \}
        \}                       
    \}   
\end{DoxyCode}
 This constraint is equivalent to $d\boldsymbol{u}_\text{m}=d\boldsymbol{u}$ which can be applied during the nonlinear solve step. We only need to overload function \hyperlink{classsolve_class_a029ece57f667fa697cb29eb482eff31b}{apply\-\_\-d\-U\-\_\-constrain }. 
\begin{DoxyCode}
\textcolor{keyword}{template} <\textcolor{keywordtype}{int} dim>
\textcolor{keywordtype}{void} initBoundValProbs<dim>::apply_dU_constrain(PETScWrappers::MPI::Vector& dU)
\{
    PETScWrappers::Vector localized\_dU(this->dU);   
  \textcolor{keywordflow}{for}(\textcolor{keywordtype}{unsigned} \textcolor{keywordtype}{int} i=0;i<constrain\_index[0].size();i++)\{
        localized\_dU(constrain\_index[0][i])=localized\_dU(constrain\_index[1][i]);
    \}
    dU = localized\_dU;
\}
\end{DoxyCode}
 Also use another \hyperlink{}{nonlinear solver function } which take a extra argument {\bfseries vector\-Type\& d\-U}.\hypertarget{growth_results}{}\section{Results}\label{growth_results}
We present few results here by using the following Parameter file\-: 
\begin{DoxyCode}
\textcolor{preprocessor}{#parameters file}
\textcolor{preprocessor}{}\textcolor{preprocessor}{#1. Thermal modeling of cylindrical lithium ion battery during discharge cycle, Dong Hyup Jeon ⇑,1, Seung
       Man Baek,2011,Energy Conversion and Management LiC6}
\textcolor{preprocessor}{}\textcolor{preprocessor}{# 2. Numerical study on lithium titanate battery thermal response under adiabatic condition,Qiujuan Sun a,
       Qingsong Wanga 2015, Energy Conversion and Management}
\textcolor{preprocessor}{}\textcolor{preprocessor}{#3.white Theoretical Analysis of Stresses in a Lithium Ion Cell    }
\textcolor{preprocessor}{}\textcolor{preprocessor}{#4 A Computational Model of the Mechanical Behavior within Reconstructed LixCoO2 Li-ion Battery Cathode
       Particles, Veruska Malavé, J.R. Berger, EA}
\textcolor{preprocessor}{}\textcolor{preprocessor}{#5. A pseudo three-dimensional electrochemicalethermal model of a prismatic LiFePO4 battery during
       discharge process}
\textcolor{preprocessor}{}\textcolor{preprocessor}{#6. solid diffusion Single-Particle Model for a Lithium-Ion Cell: Thermal Behavior, Meng Guo,a Godfrey
       Sikha,b,* and Ralph E. Whitea}
\textcolor{preprocessor}{}\textcolor{preprocessor}{#global parameters}
\textcolor{preprocessor}{}

\textcolor{preprocessor}{#declare problem setting}
\textcolor{preprocessor}{}subsection Problem
set print\_parameter = \textcolor{keyword}{true}
set dt = 0.5
set totalTime = 370
set step\_load = \textcolor{keyword}{true}

set first\_domain\_id = 1
set active\_material\_id=1
set electrolyte\_id=2
set current\_collector\_id=3
set binder\_id=3 
set solid\_id=4

set active\_material\_fe=0
set electrolyte\_fe=1
set current\_collector\_fe=2
set binder\_fe=2\textcolor{comment}{//not necessary}
set solid\_fe=3


\textcolor{preprocessor}{#directory}
\textcolor{preprocessor}{}set mesh = /home/wzhenlin/workspace/batteryCode/mesh/2D/2D\_R10\_labeled.inp
set output\_directory = output/
set snapshot\_directory = snapshot/ 
\textcolor{preprocessor}{#FEM}
\textcolor{preprocessor}{}set volume\_quadrature = 4 
set face\_quadrature = 3 
\textcolor{preprocessor}{#applied current}
\textcolor{preprocessor}{}set IpA = -350 
end

\textcolor{preprocessor}{# some useful geometry information beforehand}
\textcolor{preprocessor}{}subsection Geometry
set X\_0 = -13
set Y\_0 = 0
set Z\_0 = 0
set X\_end = 13 
set Y\_end = 153
set Z\_end = 0

set electrode\_Y1 = 76
set electrode\_Y2 = 99
set currentCollector\_Y1 = 16
set currentCollector\_Y2 = 144
set particle\_R = 10
set particle\_number = 15
end

\textcolor{preprocessor}{# initial condition}
\textcolor{preprocessor}{}subsection Initial condition
set c\_li\_max\_neg = 28.7e-3
set c\_li\_max\_pos = 37.5e-3
set c\_li\_100\_neg = 0.915
set c\_li\_100\_pos = 0.022
set c\_li\_0\_neg = 0.02
set c\_li\_0\_pos = 0.98
set c\_li\_plus\_ini = 1.0e-3
set T\_0 = 298
end

\textcolor{preprocessor}{# parameter for elastiticy equations}
\textcolor{preprocessor}{}subsection Elasticity
set youngsModulus\_Al = 70e3 #cite5
set youngsModulus\_Cu = 120e3 #cite5 
set youngsModulus\_binder = 2.3e3
set youngsModulus\_s\_neg = 12e3 #cite 3 in porosity paper
set youngsModulus\_s\_pos = 370e3 #cite4
set youngsModulus\_sep = 0.5e3 #cite 3 in porosity paper

set nu\_Al = 0.35
set nu\_Cu = 0.34
set nu\_binder = 0.3 
set nu\_sep = 0.35 #cite 3 in porosity paper
set nu\_s\_neg = 0.3 #cite 3 in porosity paper
set nu\_s\_pos = 0.2 #cite4 

set omega\_c = 3.5 #not used but 3.5 is from linear coff in porosity paper
\textcolor{preprocessor}{#following from cite 3 in porosity paper}
\textcolor{preprocessor}{}set omega\_t\_s\_neg = 8.62e-6 
set omega\_t\_s\_pos = 4.06e-6
set omega\_t\_sep = 13.32e-5
set omega\_t\_Al = 23.6e-6
set omega\_t\_Cu = 17e-6
set omega\_t\_binder = 190e-6 
end

\textcolor{preprocessor}{#parameter for fluid equations}
\textcolor{preprocessor}{}subsection Fluid
set viscosity = 1.0 #original data is 1.0e-9, but scaled to 1.0
set youngsModulus\_mesh = 1.0e3
set nu\_mesh = 0.3
end

\textcolor{preprocessor}{# parameter for electro-chemo equations}
\textcolor{preprocessor}{}subsection ElectroChemo #cite6
set sigma\_s\_neg = 1.5e8
set sigma\_s\_pos = 0.5e8
set sigma\_Al = 3.5e9
set sigma\_Cu = 6e9
set sigma\_binder =1e8

set t\_0 = 0.2
set D\_li\_neg = 5e-1 #cite 40 in porosity paper in code use expression cite 3 
set D\_li\_pos = 1.0e-1 #cite3

\textcolor{preprocessor}{# parameter for Butler-Volmer equations at ElectricChemo class}
\textcolor{preprocessor}{}set F = 96485.3329
set Rr = 8.3144598
set alpha\_a = 0.5
set alpha\_c = 0.5
set k\_neg = 0.8 #to match porosity paper
set k\_pos = 0.8 #to match porosity paper
set c\_max\_neg = 28.7e-3
set c\_max\_pos = 37.5e-3

end

\textcolor{preprocessor}{# parameter for thermal equations}
\textcolor{preprocessor}{}subsection Thermal
set lambda\_s\_neg = 5e6 #cite1
set lambda\_s\_pos = 1.85e6  #cite1
set lambda\_e = 0.33e6  #cite2
set lambda\_sep = 1e6  #cite1
set lambda\_Al = 160e6  #cite5
set lambda\_Cu = 400e6  #cite5
set lambda\_binder = 0.19e6 #PVDF

set density\_s\_neg = 5.03e-15  #cite1
set density\_s\_pos = 2.29e-15  #cite1
set density\_sep = 1.2e15  #cite1
set density\_e = 1e-15  #cite2
set density\_Al = 2.7e-15  #cite5
set density\_Cu = 8.9e-15  #cite5
set density\_binder = 1.78e-15  #cite5

set Cp\_s\_neg = 0.7e12  #cite1
set Cp\_s\_pos = 1.17e12  #cite1
set Cp\_e = 1978.16e12  #cite2
set Cp\_sep = 700e12  #cite1
set Cp\_Al = 903e12  #cite5
set Cp\_Cu = 385e12  #cite5
set Cp\_binder = 700e12 

set h = 5 #Heat transfer coefficient
end
            
                    
\textcolor{preprocessor}{#==============================================================================}
\textcolor{preprocessor}{}\textcolor{preprocessor}{# parameters reserved for deal.ii first level code:}
\textcolor{preprocessor}{}\textcolor{preprocessor}{#nonLinear\_method : classicNewton}
\textcolor{preprocessor}{}\textcolor{preprocessor}{#solver\_method (direct) : PETScsuperLU, PETScMUMPS}
\textcolor{preprocessor}{}\textcolor{preprocessor}{#solver\_method (iterative) : PETScGMRES PETScBoomerAMG}
\textcolor{preprocessor}{}\textcolor{preprocessor}{#relative\_norm\_tolerance, absolute\_norm\_tolerance, max\_iterations}
\textcolor{preprocessor}{}\textcolor{preprocessor}{#}
\textcolor{preprocessor}{}subsection Nonlinear\_solver
        set nonLinear\_method = classicNewton
        set relative\_norm\_tolerance = 1.0e-16
        set absolute\_norm\_tolerance = 1.0e-7
        set max\_iterations = 60
end
                        
subsection Linear\_solver
        set solver\_method = PETScsuperLU
        set system\_matrix\_symmetricFlag = \textcolor{keyword}{false} # \textcolor{keywordflow}{default} is \textcolor{keyword}{false}
end
\end{DoxyCode}
  

 

  The complete implementaion can be found at \href{https://github.com/mechanoChem/mechanoChemFEM/tree/example/Battery%20model%20at%20particle%20scale}{\tt Github}. 